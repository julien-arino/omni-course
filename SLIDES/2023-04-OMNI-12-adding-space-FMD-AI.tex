\documentclass[aspectratio=43]{beamer}

%\documentclass[handout]{beamer}
%% To make 4 per page
%\usepackage{pgfpages}
%\mode<handout>{\setbeamercolor{background canvas}{bg=white}}
%\pgfpagesuselayout{4 on 1}[letterpaper,landscape]%,border shrink=5mm]

\usetheme{default}
\usepackage{bm}
\usepackage{colortbl}

\usepackage{amsmath,amssymb,amsthm}
%\usepackage{subfigure}


% Fields and the like
\def\IC{\mathbb{C}}
\def\IF{\mathbb{F}}
\def\II{\mathbb{I}}
\def\IM{\mathbb{M}}
\def\IN{\mathbb{N}}
\def\IP{\mathbb{P}}
\def\IR{\mathbb{R}}
\def\IZ{\mathbb{Z}}

% Bold lowercase
\def\ba{\mathbf{a}}
\def\bb{\mathbf{b}}
\def\bc{\mathbf{c}}
\def\bd{\mathbf{d}}
\def\be{\mathbf{e}}
\def\bf{\mathbf{f}}
\def\bh{\mathbf{h}}
\def\bi{\mathbf{i}}
\def\bj{\mathbf{j}}
\def\bk{\mathbf{k}}
\def\bn{\mathbf{n}}
\def\bp{\mathbf{p}}
\def\br{\mathbf{r}}
\def\bs{\mathbf{s}}
\def\bu{\mathbf{u}}
\def\bv{\mathbf{v}}
\def\bw{\mathbf{w}}
\def\bx{\mathbf{x}}
\def\by{\mathbf{y}}
\def\bz{\mathbf{z}}

% Bold capitals
\def\bB{\mathbf{B}}
\def\bD{\mathbf{D}}
\def\bF{\mathbf{F}}
\def\bG{\mathbf{G}}
\def\bI{\mathbf{I}}
\def\bL{\mathbf{L}}
\def\bN{\mathbf{N}}
\def\bR{\mathbf{R}}
\def\bS{\mathbf{S}}
\def\bT{\mathbf{T}}
\def\bX{\mathbf{X}}

% Bold numbers
\def\b0{\mathbf{0}}

% Bold greek
\bmdefine{\bmu}{\bm{\mu}}
\def\bphi{\bm{\phi}}
\def\bvarphi{\bm{\varphi}}

% Bold red sentence
\def\boldred#1{{\color{red}\textbf{#1}}}
\def\defword#1{{\color{orange}\textbf{#1}}}

% Caligraphic letters
\def\A{\mathcal{A}}
\def\B{\mathcal{B}}
\def\C{\mathcal{C}}
\def\D{\mathcal{D}}
\def\E{\mathcal{E}}
\def\F{\mathcal{F}}
\def\G{\mathcal{G}}
\def\I{\mathcal{I}}
\def\L{\mathcal{L}}
\def\M{\mathcal{M}}
\def\P{\mathcal{P}}
\def\R{\mathcal{R}}
\def\S{\mathcal{S}}
\def\T{\mathcal{T}}
\def\U{\mathcal{U}}
\def\V{\mathcal{V}}

% tt font for code
\def\code#1{{\tt #1}}

% Operators and special symbols
\def\nbOne{{\mathchoice {\rm 1\mskip-4mu l} {\rm 1\mskip-4mu l}
{\rm 1\mskip-4.5mu l} {\rm 1\mskip-5mu l}}}
\def\cov{\ensuremath{\mathsf{cov}}}
\def\Var{\ensuremath{\mathsf{Var}\ }}
\def\Im{\textrm{Im}\;}
\def\Re{\textrm{Re}\;}
\def\det{\ensuremath{\mathsf{det}}}
\def\diag{\ensuremath{\mathsf{diag}}}
\def\nullspace{\ensuremath{\mathsf{null}}}
\def\nullity{\ensuremath{\mathsf{nullity}}}
\def\rank{\ensuremath{\mathsf{rank}}}
\def\range{\ensuremath{\mathsf{range}}}
\def\sgn{\ensuremath{\mathsf{sgn}}}
\def\Span{\ensuremath{\mathsf{span}}}
\def\tr{\ensuremath{\mathsf{tr}}}
\def\imply{$\Rightarrow$}
\def\restrictTo#1#2{\left.#1\right|_{#2}}
\newcommand{\parallelsum}{\mathbin{\!/\mkern-5mu/\!}}

% The beamer bullet (in base colour)
\def\bbullet{\leavevmode\usebeamertemplate{itemize item}\ }

% Theorems and the like
\newtheorem{proposition}[theorem]{Proposition}
\newtheorem{property}[theorem]{Property}
\newtheorem{importantproperty}[theorem]{Property}
\newtheorem{importanttheorem}[theorem]{Theorem}
%\newtheorem{lemma}[theorem]{Lemma}
%
%\usecolortheme{orchid}
\setbeamertemplate{theorems}[numbered]
%\usecolortheme{orchid}
%\setbeamertemplate{theorems}[ams style]
%\setbeamertemplate{theorems}[numbered]

%% Listings
\usepackage{listings}
\definecolor{mygreen}{rgb}{0,0.6,0}
\definecolor{mygray}{rgb}{0.5,0.5,0.5}
\definecolor{mymauve}{rgb}{0.58,0,0.82}
\definecolor{mygold}{rgb}{1,0.843,0}
\definecolor{myblue}{rgb}{0.537,0.812,0.941}

\definecolor{lgreen}{rgb}{0.6,0.9,.6}
\definecolor{lred}{rgb}{1,0.5,.5}

\lstloadlanguages{R}
\lstset{ %
  language=R,
  backgroundcolor=\color{black!95},   % choose the background color
  basicstyle=\footnotesize\ttfamily,        % size of fonts used for the code
  breaklines=true,                 % automatic line breaking only at whitespace
  captionpos=b,                    % sets the caption-position to bottom
  commentstyle=\color{mygreen},    % comment style
  escapeinside={\%*}{*)},          % if you want to add LaTeX within your code
  keywordstyle=\color{myblue},       % keyword style
  stringstyle=\color{mygold},     % string literal style
  keepspaces=true,
  columns=fullflexible,
  tabsize=4,
}
% Could also do (in lstset)
% basicstyle==\fontfamily{pcr}\footnotesize


% Get rid of navigation stuff
\setbeamertemplate{navigation symbols}{}

% Set footline/header line
\setbeamertemplate{footline}
{%
\quad p. \insertpagenumber \quad--\quad \insertsection\vskip2pt
}
% \setbeamertemplate{headline}
% {%
% \quad\insertsection\hfill p. \insertpagenumber\quad\mbox{}\vskip2pt
% }


\makeatletter
\newlength\beamerleftmargin
\setlength\beamerleftmargin{\Gm@lmargin}
\makeatother

%%%%%%% 
%% Definitions in yellow boxes
\usepackage{etoolbox}
\setbeamercolor{block title}{use=structure,fg=structure.fg,bg=structure.fg!05!bg}
\setbeamercolor{block body}{parent=normal text,use=block title,bg=block title.bg!20!bg}

\BeforeBeginEnvironment{definition}{%
	\setbeamercolor{block title}{fg=black,bg=yellow!20!white}
	\setbeamercolor{block body}{fg=black, bg=yellow!05!white}
}
\AfterEndEnvironment{definition}{
	\setbeamercolor{block title}{use=structure,fg=structure.fg,bg=structure.fg!20!bg}
	\setbeamercolor{block body}{parent=normal text,use=block title,bg=block title.bg!50!bg, fg=black}
}
\BeforeBeginEnvironment{importanttheorem}{%
	\setbeamercolor{block title}{fg=black,bg=red!20!white}
	\setbeamercolor{block body}{fg=black, bg=red!05!white}
}
\AfterEndEnvironment{importanttheorem}{
	\setbeamercolor{block title}{use=structure,fg=structure.fg,bg=structure.fg!20!bg}
	\setbeamercolor{block body}{parent=normal text,use=block title,bg=block title.bg!50!bg, fg=black}
}
\BeforeBeginEnvironment{theorem}{%
	\setbeamercolor{block title}{fg=white,bg=red!30!black}
	\setbeamercolor{block body}{fg=white, bg=red!10!black}
}
\AfterEndEnvironment{theorem}{
	\setbeamercolor{block title}{use=structure,fg=structure.fg,bg=structure.fg!20!bg}
	\setbeamercolor{block body}{parent=normal text,use=block title,bg=block title.bg!50!bg, fg=black}
}
\BeforeBeginEnvironment{importantproperty}{%
	\setbeamercolor{block title}{fg=black,bg=red!50!white}
	\setbeamercolor{block body}{fg=black, bg=red!30!white}
}
\AfterEndEnvironment{importantproperty}{
	\setbeamercolor{block title}{use=structure,fg=structure.fg,bg=structure.fg!20!bg}
	\setbeamercolor{block body}{parent=normal text,use=block title,bg=block title.bg!50!bg, fg=black}
}



%%%%%%%%%%%%%%%%%
\usepackage{tikz}
\usetikzlibrary{shapes,arrows}
\usetikzlibrary{positioning}
\usetikzlibrary{shapes.symbols,shapes.callouts,patterns}
\usetikzlibrary{calc,fit}
\usetikzlibrary{backgrounds}
\usetikzlibrary{decorations.pathmorphing,fit,petri}
\usetikzlibrary{automata}
\usetikzlibrary{fadings}
\usetikzlibrary{patterns,hobby}

\usepackage{pgfplots}
\pgfplotsset{compat=1.6}
\pgfplotsset{ticks=none}

\usetikzlibrary{decorations.markings}
\usetikzlibrary{arrows.meta}
\tikzset{>=stealth}

\tikzstyle{cloud} = [draw, 
ellipse,
fill=red!20, 
node distance=0.87cm,
minimum height=2em]
\tikzstyle{line} = [draw, 
-latex', 
color=yellow]


% Beginning of a section
% \AtBeginSection[]{
% 	{
% 		\setbeamercolor{background canvas}{bg=orange!10}
% 		\begin{frame}[noframenumbering,plain]
% 			\framesubtitle{\nameofthepart Chapter \insertromanpartnumber \ -- \iteminsert{\insertpart}}
% 			\tableofcontents[currentsection,currentsubsection]
% 		\end{frame}
% 	\addtocounter{page}{-1}
% 	%\addtocounter{framenumber}{-1} 
% 	}
% }


%%% SLIDES COLOURING

%\usecolortheme{owl}

\setbeamerfont{frametitle}{series=\bfseries}
\setbeamercolor{frametitle}{fg=black!05,bg=black}

\setbeamerfont{framesubtitle}{size=\normalfont\tiny}
\setbeamercolor{framesubtitle}{fg=black!05}

\setbeamercolor{background canvas}{bg=black}
\setbeamercolor{normal text}{fg=black!10}



\definecolor{bottomcolour}{rgb}{0.32,0.3,0.38}
\definecolor{middlecolour}{rgb}{0.08,0.08,0.16}
\definecolor{mycolor}{rgb}{0.4,0.4, 0.4}
% Beginning of a section
\AtBeginSection[]{
	{
		%\setbeamercolor{background canvas}[vertical shading][top=bottomcolour, middle=middlecolour, bottom=black]
		\setbeamertemplate{background canvas}[vertical shading][bottom=bottomcolour,top=black!20]
    % \setbeamertemplate{background canvas}{
    %   \begin{tikzpicture}%[remember picture,overlay]
    %     \shade[top color=yellow!75!green!33,
    %     bottom color=blue!66!green!33,
    %     middle color=blue!6!green!33]
    %   \end{tikzpicture}
    % }
    \begin{frame}[noframenumbering,plain]
			\framesubtitle{\nameofthepart Chapter \insertromanpartnumber \ -- \iteminsert{\insertpart}}
			\tableofcontents[currentsection,currentsubsection]
		\end{frame}
	\addtocounter{page}{-1}
	}
}
% Beginning of a section
\AtBeginSubsection[]{
	{
		%\setbeamercolor{background canvas}[vertical shading][top=bottomcolour, middle=middlecolour, bottom=black]
		%\setbeamertemplate{background canvas}[vertical shading][bottom=bottomcolour,top=black!20]
    \setbeamertemplate{background canvas}{
      \begin{tikzpicture}%[remember picture,overlay]
        \shade[top color=yellow!75!green!33,
        bottom color=blue!66!green!33,
        middle color=blue!6!green!33]
      \end{tikzpicture}
    }
    \begin{frame}[noframenumbering,plain]
			\framesubtitle{\nameofthepart Chapter \insertromanpartnumber \ -- \iteminsert{\insertpart}}
			\tableofcontents[currentsection,currentsubsection]
		\end{frame}
	\addtocounter{page}{-1}
	}
}

% Colours for special pages
\def\extraContent{yellow!20}

%% Allow to change slide colour
%% From: https://tex.stackexchange.com/questions/8043/change-the-background-color-of-a-frame-in-beamer
\defbeamertemplate*{background canvas}{mydefault}{%
  \ifbeamercolorempty[bg]{background canvas}{}{\color{bg}\vrule width\paperwidth height\paperheight}% copied beamer default here
}
\defbeamertemplate*{background canvas}{bg}{%
  \color{lightgray!20}\vrule width\paperwidth height\paperheight% added bg color
}
\BeforeBeginEnvironment{frame}{%
  \setbeamertemplate{background canvas}[mydefault]%
}
\makeatletter
\define@key{beamerframe}{bg}[true]{%
  \setbeamertemplate{background canvas}[bg]%
}
\makeatother
% Use with
%\begin{frame}
% \frametitle{Normal}
%\end{frame} 
%\begin{frame}[bg]
% \frametitle{With bg}
%\end{frame}


%% Vertical alignment on pages
%% From: https://tex.stackexchange.com/questions/148365/how-do-i-ask-beamer-to-exactly-fill-up-a-slide
%% Turn on with
%% \stretchon
%% (outside slide), and off with
%% \stretchoff
% \def\itemsymbol{$\blacktriangleright$}
% \let\svpar\par
% \let\svitemize\itemize
% \let\svenditemize\enditemize
% \let\svitem\item
% \let\svcenter\center
% \let\svendcenter\endcenter
% \let\svcolumn\column
% \let\svendcolumn\endcolumn
% \def\newitem{\renewcommand\item[1][\itemsymbol]{\vfill\svitem[##1]}}%
% \def\newpar{\def\par{\svpar\vfill}}%
% \newcommand\stretchon{%
%   \newpar%
%   \renewcommand\item[1][\itemsymbol]{\svitem[##1]\newitem}%
%   \renewenvironment{itemize}%
%     {\svitemize}{\svenditemize\newpar\par}%
%   \renewenvironment{center}%
%     {\svcenter\newpar}{\svendcenter\newpar}%
%   \renewenvironment{column}[2]%
%     {\svcolumn{##1}\setlength{\parskip}{\columnskip}##2}%
%     {\svendcolumn\vspace{\columnskip}}%
% }
% \newcommand\stretchoff{%
%   \let\par\svpar%
%   \let\item\svitem%
%   \let\itemize\svitemize%
%   \let\enditemize\svenditemize%
%   \let\center\svcenter%
%   \let\endcenter\svendcenter%
%   \let\column\svcolumn%
%   \let\endcolumn\svendcolumn%
% }
% \newlength\columnskip
% \columnskip 0pt


\newenvironment{changemargin}[2]{%
\begin{list}{}{%
\setlength{\topsep}{0pt}%
\setlength{\leftmargin}{#1}%
\setlength{\rightmargin}{#2}%
\setlength{\listparindent}{\parindent}%
\setlength{\itemindent}{\parindent}%
\setlength{\parsep}{\parskip}%
}%
\item[]}{\end{list}}


% Make one image take up the entire slide content area in beamer,.:
% centered/centred full-screen image, with title:
% This uses the whole screen except for the 1cm border around it
% all. 128x96mm
\newcommand{\titledFrameImage}[2]{
\begin{frame}{#1}
%\begin{changemargin}{-1cm}{-1cm}
\begin{center}
\includegraphics[width=108mm,height=\textheight,keepaspectratio]{#2}
\end{center}
%\end{changemargin}
\end{frame}
}

% Make one image take up the entire slide content area in beamer.:
% centered/centred full-screen image, no title:
% This uses the whole screen except for the 1cm border around it
% all. 128x96mm
\newcommand{\plainFrameImage}[1]{
\begin{frame}[plain]
%\begin{changemargin}{-1cm}{-1cm}
\begin{center}
\includegraphics[width=108mm,height=76mm,keepaspectratio]{#1}
\end{center}
%\end{changemargin}
\end{frame}
}

% Make one image take up the entire slide area, including borders, in beamer.:
% centered/centred full-screen image, no title:
% This uses the entire whole screen
\newcommand{\maxFrameImage}[1]{
\begin{frame}[plain]
\begin{changemargin}{-1cm}{-1cm}
\begin{center}
\includegraphics[width=\paperwidth,height=\paperheight,keepaspectratio]
{#1}
\end{center}
\end{changemargin}
\end{frame}
}

% Make one image take up the entire slide area, including borders, in beamer.:
% centered/centred full-screen image, no title:
% This uses the entire whole screen
\newcommand{\maxFrameImageColor}[2]{
\begin{frame}[plain]
\setbeamercolor{normal text}{bg=#2!20}
\begin{changemargin}{-1cm}{-1cm}
\begin{center}
\includegraphics[width=\paperwidth,height=\paperheight,keepaspectratio]
{#1}
\end{center}
\end{changemargin}
\end{frame}
}



\title{Adding space to FMD and AI models}
\author{Julien Arino}
\date{April 2023}


\begin{document}
%\stretchon

% The title page
\begin{frame}[noframenumbering,plain]
  \titlepage
\end{frame}
\addtocounter{page}{-1}


\begin{frame}
\begin{itemize}
  \item We now consider spatial spread
\end{itemize}
\end{frame}




%%%%%%%%%%%%%%%%%%%
%%%%%%%%%%%%%%%%%%%
%%%%%%%%%%%%%%%%%%%
%%%%%%%%%%%%%%%%%%%
\section[Brief history]{A brief history of metapopulations}

\frame{\frametitle{What are metapopulations?}
Metapopulations are \emph{populations} of \emph{populations}.
\vfill
Two main types of metapopulation models:
\begin{itemize}
\item \emph{patch occupancy models}. Describe whether a location is \emph{occupied} by a species or not. Depends on the occupancy of neighboring or connected locations. Dynamics describes the number of occupied locations
\item Models with \emph{explicit movement}. Movement between locations is described explicitly. In each location, a set of differential equations describes the dynamics of the populations present
\end{itemize}
}

\frame{\frametitle{What is a location?}
A \emph{location} is a unit (typically geographical) within which the population is considered homogeneous
\begin{itemize}
\item city
\item region
\item country
\item but also, location where a given species lives (for example, forest, swamp, etc.)
\end{itemize}
\vfill
Locations may or may not overlap
}

%%%%%%%%%%%%%%%%%%%
%%%%%%%%%%%%%%%%%%%
\subsection[Implicit movement]{Metapopulations \`a la Levins}

\begin{frame}{A model of Richard Levins (1969)}
    R. Levins. \href{https://doi.org/10.1093/besa/15.3.237}{Some Demographic and Genetic Consequences of Environmental Heterogeneity for Biological Control}. 
    Bulletin of the Entomological Society of America \textbf{15}(3): 237-240 (1969)
    \vfill
    Cited 4,400+ times, numerous higher order ``offspring''
	\vfill
	Quickly evolved to include prey-predators or competition systems
\end{frame}

\begin{frame}{The Levins model}
Rate of change of \# of local populations $P$:
\begin{equation}
P' = \beta P \left( 1 - \frac{P}{T} \right) -\mu P
\label{eq:Levins}
\end{equation}
$\beta$ immigration rate between \emph{locations}, $T$ total number of locations and $\mu$ extinction rate of local populations
\vfill
Ecologists \& mathematicians think of patches differently. For mathematicians, typically, one place in space. To be clear, in the remainder of these slides, I will speak of \emph{locations}
\end{frame}

\begin{frame}{Metapopulations with implicit movement}
    Same philosophy as the Levins model
    \vfill
    \begin{itemize}
        \item There is a set $\P$ of locations called \emph{locations}
        \item Each location $p\in\P$ has an internal dynamics $x_p=f_p(x_p)$, where $x_p\in\IR_+^{n_p}$ and $f_p:\IR^{n_p}\to \IR^{n_p}$
        \item No flow of individuals between locations
        \item The influence of location $q\neq p$ on $p$ is described through a function $g_{qp}(x_p,x_q)$, where $x_q\in\IR^{n_q}$ and $g_p:\IR^{n_p}\times\IR^{n_q}\to\IR^{n_p}$
    \end{itemize}
    \vfill
    So the population in location $p\in\P$ has dynamics
    \begin{equation}\label{eq:metapop_implicit_mvt_generic}
        x_p' = f_p(x_p)
        +\sum\dind{q\in\P}{q\neq p}
        g_{qp}(x_p,x_q)
    \end{equation}
\end{frame}

%%%%%%%%%%%%%%%%%%%
%%%%%%%%%%%%%%%%%%%
\subsection[Explicit movement]{Metapopulations with explicit movement}

\begin{frame}{Levins-type vs Explicit movement}
	Levins model and its offspring: movement is implicit
	\[
	P' = \beta P \left( 1 - \frac{P}{T} \right) -\mu P
	\]
	$\beta$ immigration rate between locations incorporates geography
	\vfill
	Sometimes we have explicit movement information or want to incorporate known spatial information $\implies$ models with explicit movement 
	\vfill Levin (1974) 
\end{frame}

\frame{\frametitle{Metapopulations with explicit movement}
	Split continuous space into $N$ discrete geographical locations (\emph{ptatches})
	\vfill
	Each location contains \textbf{compartments} (homogeneous groups of individuals). E.g., preys, predators, etc. 
	\vfill
	Here, we consider a single compartment, the \emph{species of interest}, with no further compartmentalisation
	\vfill
	Individuals \emph{may} move between locations; $m_{qp}\geq 0$ rate of movement of individuals from location $p=1,\ldots,N$ to location $q=1,\ldots,N$
	\vfill
}

\frame{\frametitle{Explicit movement (focus on $P_1$)}
	\begin{center}		
	\begin{tikzpicture}[auto, %node distance = 2cm, auto,
	cloud/.style={minimum width={width("N-1")+2pt},
		draw, ellipse,fill=red!70}]
	\node [cloud] (P1) {$P_1$};
	\node [cloud, above right=of P1] (P2) {$P_2$};
	\node [cloud, right=of P1] (P3) {$P_3$};
	\node [cloud, below right=of P1] (P4) {$P_4$};
	\node [cloud, below=of P1] (P5) {$P_5$};
	\node [cloud, below left=of P1] (P6) {$P_6$};
	\node [cloud, above left=of P1] (Pk) {$P_k$};
	%% Outflows
	\path [line, very thick, blue] (P1) to [bend left = 20] node [midway, left] (TextNode) {$m_{21}$} (P2);
	\path [line, very thick, blue] (P1) to [bend left = 20] node [midway, below] (TextNode) {$m_{31}$} (P3);
	\path [line, very thick, blue] (P1) to [bend left = 20] node [pos=0.9, right] (TextNode) {$m_{51}$} (P5);
	\path [line, very thick, blue] (P1) to [bend left = 20] node [midway, right] (TextNode) {$m_{61}$} (P6);
	\path [line, very thick, blue, densely dashed] (P1) to [bend left = 20] node [midway, left] (TextNode) {$m_{k1}$} (Pk);
	%% Inflows
	\path [line, very thick, red] (P2) to [bend left = 20] node [midway, right] (TextNode) {$m_{12}$} (P1);
	\path [line, very thick, red] (P4) to [bend left = 20] node [right] (TextNode) {$m_{14}$} (P1);
	\path [line, very thick, red] (P6) to [bend left = 20] node [left] (TextNode) {$m_{16}$} (P1);
	\path [line, very thick, red, densely dashed] (Pk) to [bend left = 20] node [right, pos=0.1] (TextNode) {$m_{1k}$} (P1);
	\end{tikzpicture}
	\end{center}
	\[
	P_1' = {\red \sum_{\substack{j=1\\j\neq 1}}^{N}m_{1j}P_j}
	{\blue -P_1\sum_{\substack{j=1\\j\neq 1}}^{N}m_{j1}}
	\]
	or
	\[
	P_1' = \sum_{j=1}^{N} m_{1j}P_j
	\textrm{ assuming }
	m_{11}=-\sum_{\substack{j=1\\j\neq 1}}^{N} m_{j1}
	\]
}


%%%%%%%%%%%%%%%%%%%
%%%%%%%%%%%%%%%%%%%
%%%%%%%%%%%%%%%%%%%
%%%%%%%%%%%%%%%%%%%
\section{The general context}

%%%%%%%%%%%%%%%%%%%
%%%%%%%%%%%%%%%%%%%
\subsection{The graph setting}


\begin{frame}\frametitle{Graph setting}
Suppose
\begin{itemize} 
\item $|\P|$ locations, vertices in a (directed) graph $\G$
\item Each location contains a certain number of compartments belonging to a common set $\C$ of compartments
\item Arcs of $\G$ represent the possibility for a given compartment to move between two locations; any two locations are connected by a maximum of $|\C|$ edges
\end{itemize}
\vfill
Graph is a digraph: movement is not always symmetric
\end{frame}


\begin{frame}
$\G=(\P,\A)$ is multi-digraph, where 
\begin{itemize}
	\item $\P$ is the set of vertices (locations)
	\item $\A$ is the set of arcs, i.e., an ordered multiset of pairs of elements of $\P$
\end{itemize}
\vfill
Any two vertices $X,Y\in\P$ are connected by at most $|\C|$ arcs from $X$ to $Y$ and at most $|\C|$ arcs from $Y$ to $X$
\vfill
Because there are $|\C|$ compartments and movements are compartment-specific, we also define, for all $c\in\C$, $\P_c$ and $\A_c$ as well as the compartment-specific digraphs $\G^c=(\P_c,\A_c)$
\end{frame}


\frame{\frametitle{Connection matrix}
For a given compartment $c\in\C$, a \emph{connection matrix} can be associated to the digraph $\G_c$
\vfill
This is the \textbf{adjacency matrix} of $\G_c$, but we emphasize the reason why we use $\G_c$ by using the term \emph{connection}
\vfill
Choosing an ordering of elements of $\P$, the $(i,j)$ entry of the $|\P|\times|\P|$-matrix $\N_c=\N_c(\G_c)$ is one if $R^c(P_i,P_j)$ and zero otherwise, i.e., if $P_i$ has no direct access to $P_j$
\vfill
For convenience, the ordering of the locations is generally assumed the same for all compartments
}

\frame{\frametitle{Strongly connected multi-digraph}
\begin{definition}[Strongly connected components]
For a given compartment $s$, the \textbf{strongly connected components} (or \textbf{strong components}, for short) are such that, for all locations $X,Y$ in a strong component, compartment $s$ in $X$ has access to $Y$
\end{definition}
\vfill
\begin{definition}[Strong connectedness for a compartment]
The multi-digraph is strongly connected for compartment $c$ if all locations belong to the same strong component of $\G_c$
\end{definition}
}

\frame{\frametitle{Srong connectedness and irreducibility}
\begin{definition}[Reducible/irreducible matrix]
	\label{def:reducible_irreducible_matrix}
	A matrix $A$ is \textbf{reducible} if there exists a permutation matrix $P$ such that $P^TAP$ is block upper triangular. A matrix that is not reducible is \textbf{irreducible}
\end{definition}
\vfill
Matrix $A\in\IF^{n\times n}$ is irreducible if for all $i,j=1,\ldots,n$, there exists $k$ such that $a_{ij}^k>0$, where $a_{ij}^k$ is the $(i,j)$-entry in $A^k$
\vfill
\begin{theorem}
	Strong connectedness $\Leftrightarrow$ \textbf{irreducibility} of the connection matrix $\mathcal{C}_c$
\end{theorem}
}



%%%%%%%%%%%%%%%%%%%
%%%%%%%%%%%%%%%%%%%
\subsection{Generic model}

\frame{\frametitle{Notation}
\begin{itemize}
	\item $N_{cp}(t)$ number of individuals of compartment $c$ in location $p$ at time $t$
	\vfill
	\item $N_c=\left(N_{c1},\ldots,N_{c|\P|}\right)^T$ distribution of individuals of compartment $c\in\C$ among the different locations 
	\vfill
	\item $N^p=\left(N_1^p,\ldots,N_{|\P|}^p\right)^T$ composition of the population in location $p\in\P$
\end{itemize}
}

\frame{\frametitle{Metapopulation models with linear movement}
Use a linear autonomous movement operator
\vfill
Then, for a given compartment $c\in\C$ and in a given location $p\in\P$
\[
	N_{cp}'=f_{cp}(N^p)
	+\sum\dind{q\in\P}{q\neq p}m_{cpq}N_{cq} 
	-\left(\sum\dind{q\in\P}{q\neq p}m_{cqp}\right)N_{cp}
\]
where $m_{cpq}$ rate of movement of individuals in compartment $c\in\C$ from location $q\in\P$ to location $p\in\P$
}

\begin{frame}{A more compact notation}
	To make 
	\[
		N_{cp}'=f_{cp}(N^p)
		+\sum\dind{q\in\P}{q\neq p}m_{cpq}N_{cq} 
		-\left(\sum\dind{q\in\P}{q\neq p}m_{cqp}\right)N_{cp}
	\]
	more compact, denote the rate of leaving location $p$ as
	\begin{equation}
		m_{cpp} = -\sum\dind{q\in\P}{q\neq p}m_{cqp}
	\end{equation}
	Then
	\begin{equation}\label{eq:general_metapop_linear_migration}
		N_s'=f_{cp}(N^p)+\sum_{q\in\P}m_{cpq}N_{cq}
	\end{equation}
\end{frame}

\begin{frame}{Vector form of the system}
For compartment $c\in\C$,
\begin{equation}\label{eq:general_metapop}
N_c'=f(N)+\mathcal{M}_cN_c
\end{equation}
with
\begin{equation}\label{eq:movement_matrix}
\mathcal{M}_c =
\begin{pmatrix}
-\sum\limits_{k\in\P}m_{ck1} & m_{c12} & \cdots & m_{c1|\P|} \\
& & & \\
m_{c|\P|1} & m_{c|\P|2} & \cdots & -\sum\limits_{k\in\P}m_{ck|\P|}
\end{pmatrix}
\end{equation}
\end{frame}


%%%%%%%%%%%%%%%%%%
%%%%%%%%%%%%%%%%%%
\subsection{The movement matrix}

\begin{frame}{Definitions and notation for matrices}
\begin{itemize}
	\item $M\in\mathbb{R}^{n\times n}$ a square matrix with entries denoted $m_{ij}$
	\vfill
	\item $M\geq\mathbf{0}$ if $m_{ij}\geq 0$ for all $i,j$ (could be the zero matrix); $M>\mathbf{0}$ if $M\geq\mathbf{0}$ and $\exists i,j$ with $m_{ij}>0$; $M\gg\mathbf{0}$ if $m_{ij}>0$ $\forall i,j=1,\ldots,n$. Same notation for vectors
	\vfill
	\item $\sigma(M)=\{\lambda\in\mathbf{C}; M\lambda=\lambda\mathbf{v}, \mathbf{v}\neq\mathbf{0}\}$ \textbf{spectrum} of $M$
	\vfill
	\item $\rho(M)=\max_{\lambda\in\sigma(M)}\{|\lambda|\}$ \textbf{spectral radius}
	\vfill
	\item $s(M)=\max_{\lambda\in\sigma(M)}\{\Re(\lambda)\}$ \textbf{spectral abscissa} (or \textbf{stability modulus})
	\vfill
	\item $M$ is an \textbf{M-matrix} if it is a \textbf{Z-matrix} ($m_{ij}\leq 0$ for $i\neq j$) and $M = s\mathbb{I}-A$, with $A\geq 0$ and $s\geq \rho(A)$
\end{itemize}
\end{frame}

\begin{frame}{The movement matrix}
	The matrix
	\begin{equation}\tag{\ref{eq:movement_matrix}}
		\mathcal{M}_c =
		\begin{pmatrix}
		-\sum\limits_{k\in\P}m_{ck1} & m_{c12} & \cdots & m_{c1|\P|} \\
		& & & \\
		m_{c|\P|1} & m_{c|\P|2} & \cdots & -\sum\limits_{k\in\P}m_{ck|\P|}
		\end{pmatrix}
	\end{equation}
	is the \textbf{movement matrix}
	\vfill
	It plays an extremely important role in the analysis of metapopulation systems, so we'll spend some time discussing its properties
	\vfill
$\mathcal{M}_c$ describes
\begin{itemize} 
\item existence of connections
\item when they exist, their ``intensity''
\end{itemize}
\end{frame}

\begin{frame}{Properties of the movement matrix $\M$}
	First, remark $-\M_c$ is a Laplacian matrix (cf. Michael's course)
	\vfill
	\begin{lemma}\label{lemma:behaviour_Mc}
		\begin{enumerate}
			\item $0\in\sigma(\M)$ corresponding to left e.v. $\11^T$ \hfill[$\sigma$ spectrum]
			\item $-\M$ is a singular M-matrix
			\item $0=s(\M)\in\sigma(\M)$ \hfill[$s$ spectral abscissa]
			\item If $\M$ irreducible, then $s(\M)$ has multiplicity 1
		\end{enumerate}
	\end{lemma}
	\vfill
	For complete proof of Lemma~\ref{lemma:behaviour_Mc} and Proposition~\ref{prop:behaviour_Mc} (next page), see \href{http://dx.doi.org/10.1007/s11538-019-00593-1}{Arino, Bajeux \& Kirkland, BMB 2019}
\end{frame}

\begin{frame}
	\begin{proposition}[$D$ a diagonal matrix]
		\label{prop:behaviour_Mc}
		\begin{enumerate}
			\item $s(\M+d\II)=d$, $\forall d\in\IR$
			\item $s(\M+D)\in\sigma(\M+D)$ associated to $\bv>\b0$. If $\M$ irreducible, $s(\M+D)$ has  multiplicity 1 and is associated to $\bv\gg \b0$
			\item If $\diag (D)\gg \b0$, then $D-\M$ invertible M-matrix and $(D-\M)^{-1}> \b0$
			\item $\M$ irreducible and $\diag(D)>\b0$ $\Longrightarrow$ $D-\M$ nonsingular irreducible M-matrix and $(D-\M)^{-1}\gg \b0$		
		\end{enumerate}
	\end{proposition}
\end{frame}


%%%%%%%%%%%%%%%%%
%%%%%%%%%%%%%%%%%
\subsection{Behaviour of the mobility component}
\frame{\frametitle{Behaviour of the mobility component}
Assume no within-location dynamics, just movement. Then \eqref{eq:general_metapop} takes the form
\begin{equation}\label{eq:mobility_compartment}
	N_c' = \mathcal{M}_cN_c
\end{equation}
\vfill
\begin{theorem}\label{th:GAS_metapop1}
For a given compartment $c\in\C$, suppose that the movement matrix $\M_c$ is irreducible. Then for any $N_c(0)>0$, \eqref{eq:mobility_compartment} satisfies
\[
\lim_{t\to\infty}N_c(t)=N_c^\star\gg 0
\]
\end{theorem}
\vfill
Note that $N_c^\star$ depends on $\nbOne^TN_c(0)$
}



\begin{frame}{Reduction to total population per location}
Let 
\[
	T_p = \sum_{c\in\C}N_{cp}
\]
be the total population in location $p$
\vfill
It is often posssible to obtain, in each location $p\in\P$, an equation for the evolution of the total population that takes the form
\begin{equation}\label{eq:dT_total_population}
	T_p' = 
	D_p(T_p)+\sum_{c\in\C}\sum_{q\in\P}m_{cpq}N_{cq}
\end{equation}
where $D_p(T_p)$ describes the demography in location $p$
\end{frame}

\begin{frame}{Nature of the demography}
	Most common types of demographic functions
	\begin{itemize}
		\item $D_p(T_p)=b_p-d_pT_p$ (asymptotically constant population)
		\item $D_p(T_p)=b_pT_p-d_pT_p$ 
		\item $D_p(T_p)=d_pT_p-d_pT_p=0$ (constant population)
		\item $D_p(T_p)=r_pT_p(1-T_p/K_p)$ (logistic demography)
	\end{itemize}
	\vfill
	In what follows, assume 
	\begin{equation}\label{eq:demography_function}
		D_p(T_p)=b_p-d_pT_p
	\end{equation}
\end{frame}

\begin{frame}{Vector / matrix form of the equation}
Assuming demography is of the form \eqref{eq:demography_function}, write \eqref{eq:dT_total_population} in vector form
\begin{equation}\label{sys:dN_general} 
	\mathbf{T}'=\mathbf{b}-\bd\bT+\sum_{c\in\C}\mathcal{M}_c\bN_{c}	
\end{equation}
where 
\begin{itemize}
	\item $\mathbf{b}=(b_1,\ldots,b_{|\mathcal{P}|})^T\in\IR^{|\P|}$
	\item $\mathbf{T}=(T_1,\ldots,T_{|\mathcal{P}|})^T\in\IR^{|\P|}$
	\item $\mathbf{N}=(N_{c1},\ldots,N_{c{|\mathcal{P}|}})^T\in\IR^{|\P|}$
	\item $\bd=\mathsf{diag}\left(d_1,\ldots,d_{|\mathcal{P}|}\right)\in\IR^{|\mathcal{P}|\times|\mathcal{P}|}$
	\item $\mathcal{M}_c\in\IR^{|\mathcal{P}|\times|\mathcal{P}|}$
\end{itemize}

\end{frame}


\begin{frame}{The nice case}
Suppose movement rates \textbf{equal for all compartments}, i.e.,
$$
\M_c\equiv\M
$$
\vfill
(stronger than Definition~\ref{def:mvt_similar_for_all_compartments}, which only requires zero/nonzero patterns in all $\M_c$, $c\in\C$, to be the same)
\vfill
Then
\begin{align}
\bT' &= \bb-\bd\bT+\M\sum_{c\in\C}\bN_{c} \nonumber \\
&= \bb-\bd\bT+\M\bT \label{sys:pSLIRS_toy_dN}
\end{align}
\end{frame}


\begin{frame}{Equilibria}
$$
\begin{aligned}
\bT'=\b0 &\Leftrightarrow \mathbf{b}-\bd\mathbf{T}+\mathcal{M}\mathbf{T}=\mathbf{0} \\
&\Leftrightarrow (\bd-\mathcal{M})\mathbf{T}=\mathbf{b} \\
&\Leftrightarrow \mathbf{T}^\star=(\bd-\mathcal{M})^{-1}\mathbf{b}
\end{aligned}
$$
given, of course, that $\bd-\mathcal{M}$ (or, equivalently, $\mathcal{M}-\bd$) is invertible.. 
\vfill
Is it?
\end{frame}

\begin{frame}{Nonsingularity of $\mathcal{M}-\bd$}
Using the spectrum shift of Theorem~\ref{prop:behaviour_Mc}{(1)}
\[
s\left(\mathcal{M}-\min_{p\in\mathcal{P}}d_p\right)=-\min_{p\in\mathcal{P}}d_p
\]
This gives a constraint: for total population to behave well (in general, we want this), we \emph{must assume all death rates are positive}
\vfill
Assume they are (in other words, assume $\bd$ nonsingular). Then $\mathcal{M}-\bd$ is nonsingular and $\bT^\star=(\bd-\mathcal{M})^{-1}\mathbf{b}$ unique
\end{frame}


\begin{frame}{Behaviour of the total population}
\framesubtitle{Equal irreducible movement case}
$\mathbf{T}^\star=(\bd-\mathcal{M})^{-1}\mathbf{b}$ attracts solutions of
$$
\mathbf{T}'=\mathbf{b}-\bd\mathbf{T}+\mathcal{M}\mathbf{T}=:f(\mathbf{T})
$$
\vfill
Indeed, we have
$$
Df=\mathcal{M}-\bd
$$
\vfill
Since we now assume that $\bd$ is nonsingular, we have by Theorem~\ref{prop:behaviour_Mc}{(1)} that $s(\mathcal{M}-\min_{p\in\mathcal{P}}d_p)=-\min_{p\in\mathcal{P}}d_p<0$
\vfill
$\mathcal{M}$ irreducible $\rightarrow$ $\mathbf{T}^\star\gg 0$ (provided $\mathbf{b}>\mathbf{0}$, of course)
\end{frame}









%%%%%%%%%%%%%%%%%%%
%%%%%%%%%%%%%%%%%%%
%%%%%%%%%%%%%%%%%%%
%%%%%%%%%%%%%%%%%%%
\section{Diseases in metapopulations}



%%%%%%%%%%%%%%%%%%%
%%%%%%%%%%%%%%%%%%%
\subsection{A few sample models}


\frame{\frametitle{The toy SLIRS model in patches}
\begin{center}
	\begin{tikzpicture}[scale=1, every node/.style={transform shape},
		auto, %node distance = 2cm, auto,
		box/.style={minimum width={width("N-1")+2pt},
			draw, rectangle,fill=red!70}]
		%% States
		\node [box] at (0,2) (S) {$S$};
		\node [box] at (2,2) (L) {$L$};
		\node [box] at (4,2) (I) {$I$};
		\node [box] at (6,2) (R) {$R$};
		%% Flows
		\path [line, thick] (-1.5,2) to node [midway, above] (TextNode) {$B(N)$} (S);
		\path [line, thick] (S) to node [midway, above, sloped] (TextNode) {$\Phi$} (L);
		\path [line, thick] (L) to node [midway, above] (TextNode) {$\varepsilon L$} (I);
		\path [line, thick] (I) to node [sloped, midway, above] (TextNode) {$\gamma I$} (R);
		\draw [>=latex,->, thick, rounded corners] (R) -- +(0,0.75) -- ++(-6,0.75) node[midway, above] {$\nu R$} -- (S);
		\path [line, thick] (S) to node [midway, right] (TextNode) {$dS$} ++(0,-1);
		\path [line, thick] (L) to node [midway, right] (TextNode) {$dL$} ++(0,-1);
		\path [line, thick] (I) to node [midway, right] (TextNode) {$(d+\delta)I$} ++(0,-1);
		\path [line, thick] (R) to node [midway, right] (TextNode) {$dR$} ++(0,-1);
	\end{tikzpicture}
\end{center}
\begin{subequations}\label{sys:SLIRS}
\begin{align}
S' &= \B(N)+\nu R-\Phi-dS \\
L' &= \Phi-(\varepsilon+d)L\\
I' &= \varepsilon L-(\gamma+d+\delta)I\\
R' &= \gamma I-(\nu+d)R
\end{align}
\end{subequations}
$\Phi$ force of infection. Depends on $S,I$, possibly $N$. In general
\[
\Phi=\beta(N)\phi(S,I)
\] 

Mass action, $\Phi=\beta SI$, proportional incidence, $\Phi=\beta SI/N$
}



\frame{\frametitle{$|\mathcal{P}|$-SLIRS model}
\begin{subequations}
	\label{sys:pSLIRS}
	\begin{align}
		S_{p}' &=\mathcal{B}_p\left(N_p\right)+\nu_pR_p-\Phi_p-d_pS_p 
		\color{red}{+\textstyle{\sum_{q\in\mathcal{P}}} m_{Spq}S_{q}} 
		\label{sys:pSLIRS_dS} \\
		L_{p}' &=\Phi_p-\left( \varepsilon_{p}+d_{p}\right)L_{p}
		\color{red}{+\textstyle{\sum_{q\in\mathcal{P}}} m_{Lpq}L_{q}} 
		\label{sys:pSLIRS_dL} \\
		I_{p}' &=\varepsilon_pL_p-(\gamma_p+d_p)I_p
		\color{red}{+\textstyle{\sum_{q\in\mathcal{P}}} m_{Ipq}I_{q}} 
		\label{sys:pSLIRS_dI} \\
		R_{p}' &=\gamma _{p}I_{p}-\left(\nu_{p}+d_{p}\right)R_{p}
		\color{red}{+\textstyle{\sum_{q\in\mathcal{P}}} m_{Rpq}R_{q}} 
		\label{sys:pSLIRS_dR} 
	\end{align}
	with incidence 
	\begin{equation}
		\Phi_p=\beta_p\frac{S_pI_p}{N_p^{q_p}},\qquad q_p\in\{0,1\}
		\label{sys:pSLIRS_incidence} 
	\end{equation}
\end{subequations}	
}

\begin{frame}
\frametitle{$|\S|\;|\P|$-SLIRS (multiple species)}
$p\in\P$ and $s\in\mathcal{S}$ (a set of species)
\vfill
\begin{subequations}
	\label{sys:spSLIRS}
	\begin{align}
		S_{sp}' &= \mathcal{B}_{sp}(N_{sp})+\nu_{sp}R_{sp}-\Phi_{sp}-d_{sp}S_{sp}
		\color{red}{+\textstyle{\sum_{q\in\mathcal{P}}} m_{Sspq}S_{sq}} 
		\label{sys:spSLIRS_dS} \\
		L_{sp}' &= \Phi_{sp}-(\varepsilon_{sp}+d_{sp})L_{sp}
		\color{red}{+\textstyle{\sum_{q\in\mathcal{P}}}m_{Lspq}L_{sq}}
		\label{sys:spSLIRS_dL} \\
		I_{sp}' &= \varepsilon_{sp}L_{sp}-(\gamma_{sp}+d_{sp})I_{sp}
		\color{red}{+\textstyle{\sum_{q\in\mathcal{P}}} m_{Ispq}I_{sq}}
		\label{sys:spSLIRS_dI} \\
		R_{sp} &= \gamma _{sp}I_{sp}-(\nu_{sp}+d_{sp})R_{sp}
		\color{red}{+\textstyle{\sum_{q\in\mathcal{P}}} m_{Rspq}R_{sq}} 
		\label{sys:spSLIRS_dR} 
	\end{align}
	with incidence
	\begin{equation}
	\Phi_{sp}=\sum_{k\in\mathcal{S}}\beta_{skp}\frac{S_{sp}I_{kp}}{N_p^{q_p}},\qquad q_p\in\{0,1\}
	\label{sys:spSLIRS_incidence} 
	\end{equation}
\end{subequations}
\vfill
{\tiny
\begin{itemize}
\setlength{\itemsep}{-5pt}
\item JA, Davis, Hartley, Jordan, Miller \& PvdD. \href{https://julien-arino.github.io/assets/pdf/papers/2005_ArinoDavisHartleyJordanMillerPvdD-MMB22.pdf}{A multi-species epidemic model with spatial dynamics}. \emph{Mathematical Medicine and Biology} \textbf{22}(2):129-142 (2005)\newline 
\item JA, Jordan \& PvdD. \href{https://julien-arino.github.io/assets/pdf/papers/2007_ArinoJordanPvdD-MBS206.pdf}{Quarantine in a multi-species epidemic model with spatial dynamics}. \emph{Mathematical Biosciences} \textbf{206}(1):46-60 (2007)
\end{itemize}}
\end{frame}

\frame{
\frametitle{$|\P|^2$-SLIRS (residents-travellers)}
\begin{subequations}
	\label{sys:ppSLIRS}
	\begin{align}
		S_{pq}' =& 
		\mathcal{B}_{pq}\left(N_p^r\right)+\nu_{pq} R_{pq}-\Phi_{pq}-d_{pq}S_{pq} \color{red}{+\textstyle{\sum_{k\in\mathcal{P}}} m_{Spqk}S_{pk}} 
		\label{sys:ppSLIRS_dS} \\
		L_{pq}' =& \Phi_{pq}
		-(\varepsilon_{pq}+d_{pq})L_{pq}
		\color{red}{+\textstyle{\sum_{k\in\mathcal{P}}} m_{Lpqk}L_{pk}} 
		\label{sys:ppSLIRS_dL} \\
		I_{pq}' =& \varepsilon_{pq} L_{pq}
		-(\gamma_{pq}+d_{pq})I_{pq}
		\color{red}{+\textstyle{\sum_{k\in\mathcal{P}}} m_{Ipqk}I_{pk}} 
		\label{sys:ppSLIRS_dI} \\
		R_{pq}' =& \gamma_{pq} I_{pq}
		-(\nu_{pq}+d_{pq})R_{pq}
		\color{red}{+\textstyle{\sum_{k\in\mathcal{P}}} m_{Rpqk}R_{pk}}
		\label{sys:ppSLIRS_dR} 
	\end{align}
	with incidence
	\begin{equation}
		\Phi_{pq}=\sum_{k\in\mathcal{P}}\beta_{pqk}\frac{S_{pq}I_{kq}}{N_p^{q_q}},\qquad q_q=\{0,1\}
		\label{sys:ppSLIRS_incidence} 
	\end{equation}
\end{subequations}

\vfill
{\tiny
\begin{itemize}
%\setlength{\itemsep}{-5pt}
\item Sattenspiel \& Dietz. \href{https://doi.org/10.1016/0025-5564(94)00068-B}{A structured epidemic model incorporating geographic mobility among regions} (1995)
\item JA \& PvdD. \href{https://julien-arino.github.io/assets/pdf/papers/2003_ArinoPvdD-MPS10_correct.pdf}{A multi-city epidemic model}. \emph{Mathematical Population Studies} \textbf{10}(3):175-193 (2003)
\item JA \& PvdD. \href{https://julien-arino.github.io/assets/pdf/papers/2003_ArinoPvdD-LNCIS294.pdf}{The basic reproduction number in a multi-city compartmental epidemic model}. In \emph{Positive Systems} (2003)
\end{itemize}}
}


\frame{\frametitle{Steps for an analysis}
\textbf{Basic steps}
\begin{enumerate}
\item Well-posedness of the system
\item Existence of disease free equilibria (DFE)
\item Computation of a reproduction number $\R_0$, study local asymptotic stability of DFE
\item If DFE unique, prove global asymptotic stability when $\R_0<1$
\end{enumerate}
\textbf{Additional steps}
\begin{enumerate}\setcounter{enumi}{4}
\item Existence of \emph{mixed} equilibria, with some locations at DFE and others with disease
\item Computation of some bounds on $\R_0$
\item EEP and its LAS \& GAS properties
\end{enumerate}
$\ldots$
}

\begin{frame}{Analysis -- Toy system}
For simplicity, consider $|\mathcal{P}|$-SLIRS with $\mathcal{B}_p(N_p)=\mathcal{B}_p$
\begin{subequations}
	\label{sys:pSLIRS_toy}
	\begin{align}
		S_{p}' &=\mathcal{B}_p-\Phi_p-d_pS_p+\nu_pR_p
		+\textstyle{\sum_{q\in\mathcal{P}}} m_{Spq}S_{q} 
		\label{sys:pSLIRS_toy_dS} \\
		L_{p}' &=\Phi_p-\left( \varepsilon_{p}+d_{p}\right)L_{p}
		+\textstyle{\sum_{q\in\mathcal{P}}} m_{Lpq}L_{q} 
		\label{sys:pSLIRS_toy_dL} \\
		I_{p}' &=\varepsilon_pL_p-(\gamma_p+d_p)I_p
		+\textstyle{\sum_{q\in\mathcal{P}}} m_{Ipq}I_{q} 
		\label{sys:pSLIRS_toy_dI} \\
		R_{p}' &=\gamma _{p}I_{p}-\left(\nu_{p}+d_{p}\right)R_{p}
		+\textstyle{\sum_{q\in\mathcal{P}}} m_{Rpq}R_{q}
		\label{sys:pSLIRS_toy_dR} 
	\end{align}
	with incidence
	\begin{equation}
		\Phi_p=\beta_p\frac{S_pI_p}{N_p^{q_p}},\qquad q_p\in\{0,1\}
		\label{sys:pSLIRS_toy_incidence} 
	\end{equation}			
\end{subequations}
\vfill
System of $4|\mathcal{P}|$ equations
\end{frame}

\begin{frame}{Don't panic: size is not that bad..}
System of $4|\mathcal{P}|$ equations !!!
\vfill
However, a lot of structure: 
\begin{itemize}
	\item $|\mathcal{P}|$ \emph{copies} of individual units, each comprising 4 equations
	\item Dynamics of individual units well understood
	\item Coupling is linear
\end{itemize}
\vfill
$\implies$ Good case of large-scale system
\vfill
(matrix analysis is your friend)

\end{frame}


% %%%%%%%%%%%%%%%%%%%
% %%%%%%%%%%%%%%%%%%%
% %%%%%%%%%%%%%%%%%%%
% %%%%%%%%%%%%%%%%%%%
% %%%%%%%%%%%%%%%%%%%
% %%%%%%%%%%%%%%%%%%%
% \subsection{Well-posedness}

\frame{\frametitle{Existence and uniqueness}
\begin{itemize}
\item
Existence and uniqueness of solutions classic, assured by good choice of birth and force of infection functions
\vfill
\item
In the cases treated later, the birth function is either constant or a linear combination of state variables
\vfill
\item May exist problems at the origin, if the force of infection is not defined there
\vfill
\item
Assumption form now on: existence and uniqueness
\end{itemize}
}



%%%%%%%%%%%%%%%%%%%
%%%%%%%%%%%%%%%%%%%
\subsection{Existence of a DFE}

\frame{\frametitle{Disease free equilibrium}
The model is at equilibrium if the time derivatives are zero
\vfill
\begin{definition}[Metapopulation DFE]
In the case of system \eqref{sys:pSLIRS_toy}, location $p\in\P$ is at a disease-free equilibrium (DFE) if $L_{p}=I_{p}=0$, and the $|\P|$-location model is at a \textbf{metapopulation DFE} if $L_{p}=I_{p}=0$ for all $p\in\P$
\end{definition}
\vfill
Here, we want to find the DFE for the $|\P|$-location model. Later, the existence of mixed equilibria, with some locations at the DFE and others at an endemic equilibrium, is considered
\vfill
(For \eqref{sys:spSLIRS}, replace $L_p$ with $L_{sp}$ and $I_p$ with $I_{sp}$, for \eqref{sys:ppSLIRS}, replace $L_p$ by $L_{pp}$ and $I_p$ by $I_{pp}$. To simplify notation, we could write $L_{\bullet}$ and $I_{\bullet}$)
}

\begin{frame}{}
Assume \eqref{sys:pSLIRS_toy} at metapopulation DFE. Then $\Phi_p=0$ and 
\begin{align*}
0 &=\mathcal{B}_p-d_pS_p+\nu_pR_p
+\textstyle{\sum_{q\in\mathcal{P}}} m_{Spq}S_{q} \\
0 &=-\left(\nu_{p}+d_{p}\right)R_{p}
+\textstyle{\sum_{q\in\mathcal{P}}} m_{Rpq}R_{q}
\end{align*}
Want to solve for $S_p,R_p$. Here, it is best (crucial in fact) to remember some linear algebra. Write system in vector form:
\begin{align*}
\mathbf{0} &=\mathbf{b}-\bd\mathbf{S}+\mathbf{\nu}\bR+\mathcal{M}^S\mathbf{S} \\
\mathbf{0} &=-\left(\mathbf{\nu}+\bd\right)\bR+\mathcal{M}^R\bR
\end{align*}
where $\mathbf{S},\bR,\mathbf{b}\in\mathbb{R}^{|\mathcal{P}|}$, $\bd,\mathbf{\nu},\mathcal{M}^S,\mathcal{M}^R$ $|\mathcal{P}|\times|\mathcal{P}|$-matrices ($\bd,\mathbf{\nu}$ diagonal)
\end{frame}

\begin{frame}{$\bR$ at DFE}
Recall second equation:
$$
\mathbf{0} =-\left(\mathbf{\nu}+\bd\right)\bR+\mathcal{M}^R\bR \Leftrightarrow (\mathcal{M}^R-\mathbf{\nu}-\bd)\bR=\mathbf{0}
$$
\vfill
So unique solution $\bR=\mathbf{0}$ if $\mathcal{M}^R-\mathbf{\nu}-\bd$ invertible
Is it?
\vfill
We have been here before! 
\vfill
From spectrum shift, $s(\mathcal{M}^R-\mathbf{\nu}-\bd)=-\min_{p\in\mathcal{P}}(\nu_p+d_p)<0$
\vfill
So, given $\mathbf{L}=\mathbf{I}=\mathbf{0}$, $\bR=\mathbf{0}$ is the unique equilibrium and
$$
\lim_{t\to\infty}\bR(t)=\mathbf{0}
$$
\vfill
$\implies$ DFE has $\mathbf{L}=\mathbf{I}=\bR=\mathbf{0}$
\end{frame}


\begin{frame}{$\mathbf{S}$ at the DFE}
DFE has $\mathbf{L}=\mathbf{I}=\bR=\mathbf{0}$ and $\mathbf{b}-\bd\mathbf{S}+\mathcal{M}^S\mathbf{S}=\mathbf{0}$, i.e.,
$$
\mathbf{S}=(\bd-\mathcal{M}^S)^{-1}\mathbf{b}
$$
\vfill
Recall: $-\mathcal{M}^S$ singular M-matrix. From previous reasoning, $\bd-\mathcal{M}^S$ has \textbf{instability modulus} shifted \emph{right} by $\min_{p\in\mathcal{P}}d_p$. So:
\begin{itemize}
\item $\bd-\mathcal{M}^S$ invertible
\item $\bd-\mathcal{M}^S$ nonsingular M-matrix
\end{itemize}
\vfill
Second point $\implies (\bd-\mathcal{M}^S)^{-1}>\mathbf{0}\implies (\bd-\mathcal{M}^S)^{-1}\mathbf{b}> \mathbf{0}$  (would have $\gg\mathbf{0}$ if $\mathcal{M}^S$ irreducible)
\vfill
So DFE makes sense with
$$
(\mathbf{S},\mathbf{L},\mathbf{I},\bR)=\left((\bd-\mathcal{M}^S)^{-1}\mathbf{b},\mathbf{0},\mathbf{0},\mathbf{0}\right)
$$
\end{frame}

%%%%%%%%%%%%%%%%%%%
%%%%%%%%%%%%%%%%%%%
%%%%%%%%%%%%%%%%%%%
%%%%%%%%%%%%%%%%%%%
%%%%%%%%%%%%%%%%%%%
%%%%%%%%%%%%%%%%%%%
\subsection{Computation of a reproduction number}

\begin{frame}{Computing the basic reproduction number $\R_0$}
Use next generation method with $\Xi=\{L_1,\ldots,L_{|\mathcal{P}|},I_1,\ldots,I_{|\mathcal{P}|}\}$, $\Xi'=\mathcal{F}-\mathcal{V}$
$$
\mathcal{F}=\left(\Phi_1,\ldots,\Phi_{|\mathcal{P}|},0,\ldots,0\right)^T
$$
$$
\mathcal{V}=
\begin{pmatrix}
\left( \varepsilon_{1}+d_{1}\right)L_{1}
-\sum\limits_{q\in\mathcal{P}} m_{L1q}L_{q} \\
\vdots \\
\left( \varepsilon_{|\mathcal{P}|}+d_{|\mathcal{P}|}\right)L_{|\mathcal{P}|}
-\sum\limits_{q\in\mathcal{P}} m_{L|\mathcal{P}|q}L_{q} \\
-\varepsilon_1L_1+(\gamma_1+d_1)I_1
-\sum\limits_{q\in\mathcal{P}} m_{I1q}I_{q} \\
\vdots \\
-\varepsilon_{|\mathcal{P}|}L_{|\mathcal{P}|}
+(\gamma_{|\mathcal{P}|}+d_{|\mathcal{P}|})I_{|\mathcal{P}|}
-\sum\limits_{q\in\mathcal{P}} m_{I|\mathcal{P}|q}I_{q}
\end{pmatrix}
$$
\end{frame}


\begin{frame}
Differentiate w.r.t. $\Xi$:
$$
D\mathcal{F}
=
\begin{pmatrix}
\dfrac{\partial\Phi_1}{\partial L_1} & \cdots &
\dfrac{\partial\Phi_1}{\partial L_{|\mathcal{P}|}} & 
\dfrac{\partial\Phi_1}{\partial I_1} & \cdots &
\dfrac{\partial\Phi_1}{\partial I_{|\mathcal{P}|}} \\
\vdots & & \vdots & \vdots & & \vdots \\
\dfrac{\partial\Phi_{|\mathcal{P}|}}{\partial L_1} & \cdots &
\dfrac{\partial\Phi_{|\mathcal{P}|}}{\partial L_{|\mathcal{P}|}} & 
\dfrac{\partial\Phi_{|\mathcal{P}|}}{\partial I_1} & \cdots &
\dfrac{\partial\Phi_{|\mathcal{P}|}}{\partial I_{|\mathcal{P}|}} \\
0 & \cdots & 0 & 0 & \cdots & 0 \\
\vdots & & \vdots & \vdots & & \vdots \\
0 & \cdots & 0 & 0 & \cdots & 0
\end{pmatrix}
$$
\end{frame}

\begin{frame}
Note that
$$
\frac{\partial\Phi_p}{\partial L_k}=\frac{\partial\Phi_p}{\partial I_k}=0
$$
whenever $k\neq p$, so
$$
D\mathcal{F}
=
\begin{pmatrix}
\mathsf{diag}\left(
\frac{\partial\Phi_1}{\partial L_1},\ldots,\frac{\partial\Phi_{|\mathcal{P}|}}{\partial L_{|\mathcal{P}|}}\right) &
\mathsf{diag}\left(
\frac{\partial\Phi_1}{\partial I_1},\ldots,\frac{\partial\Phi_{|\mathcal{P}|}}{\partial I_{|\mathcal{P}|}}\right) \\
\mathbf{0} & \mathbf{0} 
\end{pmatrix}
$$
\end{frame}

\begin{frame}{Evaluate $D\mathcal{F}$ at DFE}
	\begin{minipage}[t]{0.45\textwidth}
		If $\Phi_p=\beta_pS_pI_p$, then
		\begin{itemize}
			\item $\dfrac{\partial\Phi_p}{\partial L_p}=0$
			\item $\dfrac{\partial\Phi_p}{\partial I_p}=\beta_pS_p$
		\end{itemize}
	\end{minipage}
	\begin{minipage}[t]{0.45\textwidth}
		If $\Phi_p=\beta_p\dfrac{S_pI_p}{N_p}$, then
		\begin{itemize}
			\item $\dfrac{\partial\Phi_p}{\partial L_p}=\beta_p\dfrac{S_pI_p}{N_p^2}=0$ at DFE
			\item $\dfrac{\partial\Phi_p}{\partial I_p}=\beta_p\dfrac{S_p}{N_p}$ at DFE
		\end{itemize}
	\end{minipage}
	\vfill
In both cases, $\partial/\partial L$ block is zero so
$$
F=D\mathcal{F}(DFE)=
\begin{pmatrix}
\mathbf{0} & \mathsf{diag}\left(
\frac{\partial\Phi_1}{\partial I_1},\ldots,\frac{\partial\Phi_{|\mathcal{P}|}}{\partial I_{|\mathcal{P}|}}\right) \\
\mathbf{0} & \mathbf{0}
\end{pmatrix}
$$
\end{frame}

\begin{frame}{Compute $D\mathcal{V}$ and evaluate at DFE}
$$
V=
\begin{pmatrix}
\mathsf{diag}_p(\varepsilon_p+d_p)-\mathcal{M}^L & \mathbf{0} \\
-\mathsf{diag}_p(\varepsilon_p) & \mathsf{diag}_p(\gamma_p+d_p)-\mathcal{M}^I
\end{pmatrix}
$$
where $\mathsf{diag}_p(z_p):=\mathsf{diag}(z_1,\ldots,z_{|\mathcal{P}|})$
\vfill
Inverse of $V$ easy ($2\times 2$ block lower triangular):
$$
V^{-1}
=
\begin{pmatrix}
\left(\mathsf{diag}_p(\varepsilon_p+d_p)-\mathcal{M}^L\right)^{-1} & \mathbf{0} \\
\tilde V_{21}^{-1} & \left(\mathsf{diag}_p(\gamma_p+d_p)-\mathcal{M}^I\right)^{-1}
\end{pmatrix}
$$
where
\begin{multline*}
	\tilde V_{21}^{-1}=
	\left(\mathsf{diag}_p(\varepsilon_p+d_p)-\mathcal{M}^L\right)^{-1}\\ 
	\mathsf{diag}_p(\varepsilon_p)
	\left(\mathsf{diag}_p(\gamma_p+d_p)-\mathcal{M}^I\right)^{-1}		
\end{multline*}
\end{frame}

\begin{frame}{$\R_0$ as $\rho(FV^{-1})$}
Next generation matrix
$$
FV^{-1}=
\begin{pmatrix}
\mathbf{0} & F_{12} \\
\mathbf{0} & \mathbf{0}
\end{pmatrix}
\begin{pmatrix}
\tilde V_{11}^{-1} & \mathbf{0} \\
\tilde V_{21}^{-1} & \tilde V_{22}^{-1}
\end{pmatrix}
=
\begin{pmatrix}
F_{12}\tilde V_{21}^{-1} & F_{12}\tilde V_{22}^{-1} \\
\mathbf{0} & \mathbf{0}
\end{pmatrix}
$$
where $\tilde V_{ij}^{-1}$ is block $ij$ in $V^{-1}$. So
$$
\R_0=\rho\left(F_{12}\tilde{V}_{21}^{-1}\right)
$$
i.e.,
\begin{multline*}
	\R_0=\rho\Biggl(
		\mathsf{diag}\left(
		\frac{\partial\Phi_1}{\partial I_1},\ldots,\frac{\partial\Phi_{|\mathcal{P}|}}{\partial I_{|\mathcal{P}|}}\right)
		\left(\mathsf{diag}_p(\varepsilon_p+d_p)-\mathcal{M}^L\right)^{-1} \\ 
		\mathsf{diag}_p(\varepsilon_p)
		\left(\mathsf{diag}_p(\gamma_p+d_p)-\mathcal{M}^I\right)^{-1}
		\Biggr)			
\end{multline*}
\end{frame}

\begin{frame}{Local asymptotic stability of the DFE}
\begin{theorem}\label{th:LAS_DFE_pSLIRS_toy}
	Define $\R_0$ for the $|\mathcal{P}|$-SLIRS as 
	\begin{multline*}
		\R_0=\rho\Biggl(
			\mathsf{diag}\left(
			\frac{\partial\Phi_1}{\partial I_1},\ldots,\frac{\partial\Phi_{|\mathcal{P}|}}{\partial I_{|\mathcal{P}|}}\right)
			\left(\mathsf{diag}_p(\varepsilon_p+d_p)-\mathcal{M}^L\right)^{-1}  \\
			\mathsf{diag}_p(\varepsilon_p)
			\left(\mathsf{diag}_p(\gamma_p+d_p)-\mathcal{M}^I\right)^{-1}
			\Biggr)				
	\end{multline*}
	Then the DFE
	$$
	(\mathbf{S},\mathbf{L},\mathbf{I},\mathbf{R})=\left((\mathbf{d}-\mathcal{M}^S)^{-1}\mathbf{b},\mathbf{0},\mathbf{0},\mathbf{0}\right)
	$$
	is locally asymptotically stable if $\R_0<1$ and unstable if $\R_0>1$		
\end{theorem}
\vfill
{\setstretch{0.5}\tiny
From PvdD \& Watmough, \href{https://doi.org/10.1016/S0025-5564(02)00108-6}{Reproduction numbers and sub-threshold endemic equilibria for compartmental models of disease transmission}, \emph{Bulletin of Mathematical Biology} \textbf{180}(1-2): 29-48 (2002)}
\end{frame}

\begin{frame}{Some remarks about $\R_0$}
	The expression for $\R_0$ in Theorem~\ref{th:LAS_DFE_pSLIRS_toy} is exact
	\vfill
	However, unless you consider a very small set of locations, you will not get a closed form expression
	\vfill
	Indeed, by Theorem~\ref{prop:behaviour_Mc}{(3)} and more importantly (often $\M$ is irreducible), Theorem~\ref{prop:behaviour_Mc}{(4)}, the two inverses in $\R_0$ are likely crowded ($\gg 0$ in the irreducible case)
	\vfill
	However, numerically, this works easy unless conditioning is bad
\end{frame}



%%%%%%%%%%%%%%%%%%%
%%%%%%%%%%%%%%%%%%%
%%%%%%%%%%%%%%%%%%%
%%%%%%%%%%%%%%%%%%%
%%%%%%%%%%%%%%%%%%%
%%%%%%%%%%%%%%%%%%%
\subsection{Global stability of the DFE when $\R_0<1$}

\frame{\frametitle{The toy $|\P|$-SLIRS}
LAS results for $\R_0<1$ can sometimes be strengthened to GAS. One class of models where this works often is when the population is either constant or asymptotically constant and incidence is \emph{standard}
\vfill
\begin{theorem}\label{th:pSLIRS_GAS}
Let $\R_0$ be defined as in Theorem~\ref{th:LAS_DFE_pSLIRS_toy} and use proportional incidence $\Phi_p=\beta_pS_pI_p/N_p$. If $\R_0<1$, then the DFE of system \eqref{sys:pSLIRS_toy} is globally asymptotically stable
\end{theorem}
}


\frame{\frametitle{$|\S|\;|\P|$-SLIRS with multiple species}
In the case in which movement is equal for all compartments and there is no disease death, a comparison theorem argument can be used as in Theorem~\ref{th:pSLIRS_GAS} to show that if $\R_0<1$, then the
DFE of the $|\S|\;|\P|$-SLIRS \eqref{sys:spSLIRS} is globally asymptotically stable. 
\vfill
\begin{theorem}\label{th:R0_spSEIRS_GAS}
For system \eqref{sys:spSLIRS} with $|\S|$ species and $|\P|$ locations, with movement equal for all compartments, define $\R_0$ appropriately and use proportional incidence. 
If $\R_0<1$, then the DFE is globally asymptotically stable
\end{theorem}
}





%%%%%%%%%%%%%%%%%%
%%%%%%%%%%%%%%%%%%
%%%%%%%%%%%%%%%%%%
%%%%%%%%%%%%%%%%%%
\section{Computational considerations}

\begin{frame}[fragile]{Set up parameters}
\begin{lstlisting}
pop = c(34.017, 1348.932, 1224.614, 173.593, 93.261) * 1e+06
countries = c("Canada", "China", "India", "Pakistan", "Philippines")
T = matrix(data = 
				c(0, 1268, 900, 489, 200, 
				1274, 0, 678, 859, 150, 
				985, 703, 0, 148, 58, 
				515, 893, 144, 0, 9, 
				209, 174, 90, 2, 0), 
			nrow = 5, ncol = 5, byrow = TRUE)	
\end{lstlisting}
\end{frame}

\begin{frame}[fragile]{Work out movement matrix}
\begin{lstlisting}
p = list()
# Use the approximation explained in Arino & Portet (JMB 2015)
p$M = mat.or.vec(nr = dim(T)[1], nc = dim(T)[2])
for (from in 1:5) {
	for (to in 1:5) {
	p$M[to, from] = -log(1 - T[from, to]/pop[from])
	}
	p$M[from, from] = 0
}
p$M = p$M - diag(colSums(p$M))
\end{lstlisting}	
\end{frame}

\begin{frame}[fragile]
\begin{lstlisting}
p$P = dim(p$M)[1]
p$eta = rep(0.3, p$P)
p$epsilon = rep((1/1.5), p$P)
p$pi = rep(0.7, p$P)
p$gammaI = rep((1/5), p$P)
p$gammaA = rep((1/3), p$P)
# The desired values for R_0
R_0 = rep(1.5, p$P)
\end{lstlisting}	
\end{frame}

\begin{frame}[fragile]{Write down indices of the different state variable types}
	Save index of state variable types in state variables vector (we have to use a vector and thus, for instance, the name ``S'' needs to be defined)
\begin{lstlisting}
p$idx_S = 1:p$P
p$idx_L = (p$P+1):(2*p$P)
p$idx_I = (2*p$P+1):(3*p$P)
p$idx_A = (3*p$P+1):(4*p$P)
p$idx_R = (4*p$P+1):(5*p$P)
\end{lstlisting}	
\end{frame}

\begin{frame}[fragile]{Set up IC and time}
\begin{lstlisting}
# Set initial conditions. For example, we start with 2
# infectious individuals in Canada.
L0 = mat.or.vec(p$P, 1)
I0 = mat.or.vec(p$P, 1)
A0 = mat.or.vec(p$P, 1)
R0 = mat.or.vec(p$P, 1)
I0[1] = 2
S0 = pop - (L0 + I0 + A0 + R0)
# Vector of initial conditions to be passed to ODE solver.
IC = c(S = S0, L = L0, I = I0, A = A0, R = R0)
# Time span of the simulation (5 years here)
tspan = seq(from = 0, to = 5 * 365.25, by = 0.1)
\end{lstlisting}	
\end{frame}


\begin{frame}[fragile]{Set up $\beta$ to avoid blow up}
	Let us take $\R_0=1.5$ for patches in isolation. Solve $\R_0$ for $\beta$ 
	$$
	\beta=\frac{\R_0}{S(0)}
	\left(
	\frac{1-\pi_p}{\gamma_{Ip}}
	+\frac{\pi_p\eta_p}{\gamma_{Ap}}
	\right)^{-1}
	$$ 
\begin{lstlisting}
for (i in 1:p$P) {
	p$beta[i] = 
		R_0[i] / S0[i] * 1/((1 - p$pi[i])/p$gammaI[i] + p$pi[i] * p$eta[i]/p$gammaA[i])
}
\end{lstlisting}	
\end{frame}

\begin{frame}[fragile]{Define the vector field}
\begin{lstlisting}[language=Renhanced]
SLIAR_metapop_rhs <- function(t, x, p) {
	with(as.list(p), {
		S = x[idx_S]
		L = x[idx_L]
		I = x[idx_I]
		A = x[idx_A]
		R = x[idx_R]
		N = S + L + I + A + R
		Phi = beta * S * (I + eta * A) / N
		dS = - Phi + MS \%*\% S
		dL = Phi - epsilon * L + p$ML \%*\% L
		dI = (1 - pi) * epsilon * L - gammaI * I + MI \%*\% I
		dA = pi * epsilon * L - gammaA * A + MA \%*\% A
		dR = gammaI * I + gammaA * A + MR \%*\% R
		dx = list(c(dS, dL, dI, dA, dR))
		return(dx)
	})
}	
\end{lstlisting}
\end{frame}
	

\begin{frame}[fragile]{And now call the solver}
\begin{lstlisting}
# Call the ODE solver
sol <- ode(y = IC, 
			times = tspan, 
			func = SLIAR_metapop_rhs, 
			parms = p,
			method = "ode45")
\end{lstlisting}	
\end{frame}
	
\begin{frame}[fragile]{One little trick (case with demography)}
	Suppose demographic EP is $\mathbf{N}^\star=(\bd-\mathcal{M})^{-1}\mathbf{b}$

	Want to maintain $\mathbf{N}(t)=\mathbf{N}^\star$ for all $t$ to ignore convergence to demographic EP. Think in terms of $\mathbf{b}$:
	
	$$
	\mathbf{N}'=0\iff \mathbf{b}-\bd\mathbf{N}+\mathcal{M}\mathbf{N}=0 \iff \mathbf{b} = (\bd-\mathcal{M})\mathbf{N}
	$$
	
	So take $\mathbf{b}=(\bd-\mathcal{M})\mathbf{N}^\star$
	
	Then
	$$
	\mathbf{N}' = (\bd-\mathcal{M})\mathbf{N}^\star
	-\bd\mathbf{N}+\mathcal{M}\mathbf{N}
	$$
	and thus if $\mathbf{N}(0)=\mathbf{N}^\star$, then $\mathbf{N}'(0)=0$ and thus $\mathbf{N}'=0$ for all $t\geq 0$, i.e., $\mathbf{N}(t)=\mathbf{N}^\star$ for all $t\geq 0$
\end{frame}

\begin{frame}{Word of warning about that trick, though..}
$$
\mathbf{b}=(\bd-\mathcal{M})\mathbf{N}^\star
$$
\vfill
$\bd-\M$ has nonnegative (typically positive) diagonal entries and nonpositive off-diagonal entries
\vfill
Easy to think of situations where the diagonal will be dominated by the off-diagonal, so $\bb$ could have negative entries
\vfill
$\implies$ use this for numerics, not for the mathematical analysis
\end{frame}


%%%%%%%%%%%%%%%%%%%
%%%%%%%%%%%%%%%%%%%
%%%%%%%%%%%%%%%%%%%
%%%%%%%%%%%%%%%%%%%
\section{A few FMD models}

%%%%%%%%%%%%%%%%%%%
%%%%%%%%%%%%%%%%%%%
\subsection{Ringa \& Bauch}
%%%%%%%%%%%%%%%%%%%
\maxFrameImage{../FIGS/RingaBauch-FMD-pair-approx-cover-inverted.png}

\maxFrameImage{../FIGS/HaydonWoolhouse-FMD-UK-cover-inverted.png}

\begin{frame}{A model \`a la Levin}
    Space is implicit: count infected herds
\end{frame}

\maxFrameImage{../FIGS/Woolhouse-models-FMD-cover-inverted.png}


%%%%%%%%%%%%%%%%%%%
\maxFrameImage{../FIGS/Andronico_etal-H5N8-control-cover-inverted}
\maxFrameImage{../FIGS/Andronico_etal-H5N8-control-curve-inverted}


\begin{frame}{Conclusion}
\end{frame}


\end{document}
