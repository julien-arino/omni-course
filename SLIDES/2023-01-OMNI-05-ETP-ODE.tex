\documentclass[aspectratio=43]{beamer}

%\documentclass[handout]{beamer}
%% To make 4 per page
%\usepackage{pgfpages}
%\mode<handout>{\setbeamercolor{background canvas}{bg=white}}
%\pgfpagesuselayout{4 on 1}[letterpaper,landscape]%,border shrink=5mm]

\usetheme{default}
\usepackage{bm}
\usepackage{colortbl}

\usepackage{amsmath,amssymb,amsthm}
%\usepackage{subfigure}


% Fields and the like
\def\IC{\mathbb{C}}
\def\IF{\mathbb{F}}
\def\II{\mathbb{I}}
\def\IM{\mathbb{M}}
\def\IN{\mathbb{N}}
\def\IP{\mathbb{P}}
\def\IR{\mathbb{R}}
\def\IZ{\mathbb{Z}}

% Bold lowercase
\def\ba{\mathbf{a}}
\def\bb{\mathbf{b}}
\def\bc{\mathbf{c}}
\def\bd{\mathbf{d}}
\def\be{\mathbf{e}}
\def\bf{\mathbf{f}}
\def\bh{\mathbf{h}}
\def\bi{\mathbf{i}}
\def\bj{\mathbf{j}}
\def\bk{\mathbf{k}}
\def\bn{\mathbf{n}}
\def\bp{\mathbf{p}}
\def\br{\mathbf{r}}
\def\bs{\mathbf{s}}
\def\bu{\mathbf{u}}
\def\bv{\mathbf{v}}
\def\bw{\mathbf{w}}
\def\bx{\mathbf{x}}
\def\by{\mathbf{y}}
\def\bz{\mathbf{z}}

% Bold capitals
\def\bB{\mathbf{B}}
\def\bD{\mathbf{D}}
\def\bF{\mathbf{F}}
\def\bG{\mathbf{G}}
\def\bI{\mathbf{I}}
\def\bL{\mathbf{L}}
\def\bN{\mathbf{N}}
\def\bR{\mathbf{R}}
\def\bS{\mathbf{S}}
\def\bT{\mathbf{T}}
\def\bX{\mathbf{X}}

% Bold numbers
\def\b0{\mathbf{0}}

% Bold greek
\bmdefine{\bmu}{\bm{\mu}}
\def\bphi{\bm{\phi}}
\def\bvarphi{\bm{\varphi}}

% Bold red sentence
\def\boldred#1{{\color{red}\textbf{#1}}}
\def\defword#1{{\color{orange}\textbf{#1}}}

% Caligraphic letters
\def\A{\mathcal{A}}
\def\B{\mathcal{B}}
\def\C{\mathcal{C}}
\def\D{\mathcal{D}}
\def\E{\mathcal{E}}
\def\F{\mathcal{F}}
\def\G{\mathcal{G}}
\def\I{\mathcal{I}}
\def\L{\mathcal{L}}
\def\M{\mathcal{M}}
\def\P{\mathcal{P}}
\def\R{\mathcal{R}}
\def\S{\mathcal{S}}
\def\T{\mathcal{T}}
\def\U{\mathcal{U}}
\def\V{\mathcal{V}}

% tt font for code
\def\code#1{{\tt #1}}

% Operators and special symbols
\def\nbOne{{\mathchoice {\rm 1\mskip-4mu l} {\rm 1\mskip-4mu l}
{\rm 1\mskip-4.5mu l} {\rm 1\mskip-5mu l}}}
\def\cov{\ensuremath{\mathsf{cov}}}
\def\Var{\ensuremath{\mathsf{Var}\ }}
\def\Im{\textrm{Im}\;}
\def\Re{\textrm{Re}\;}
\def\det{\ensuremath{\mathsf{det}}}
\def\diag{\ensuremath{\mathsf{diag}}}
\def\nullspace{\ensuremath{\mathsf{null}}}
\def\nullity{\ensuremath{\mathsf{nullity}}}
\def\rank{\ensuremath{\mathsf{rank}}}
\def\range{\ensuremath{\mathsf{range}}}
\def\sgn{\ensuremath{\mathsf{sgn}}}
\def\Span{\ensuremath{\mathsf{span}}}
\def\tr{\ensuremath{\mathsf{tr}}}
\def\imply{$\Rightarrow$}
\def\restrictTo#1#2{\left.#1\right|_{#2}}
\newcommand{\parallelsum}{\mathbin{\!/\mkern-5mu/\!}}

% The beamer bullet (in base colour)
\def\bbullet{\leavevmode\usebeamertemplate{itemize item}\ }

% Theorems and the like
\newtheorem{proposition}[theorem]{Proposition}
\newtheorem{property}[theorem]{Property}
\newtheorem{importantproperty}[theorem]{Property}
\newtheorem{importanttheorem}[theorem]{Theorem}
%\newtheorem{lemma}[theorem]{Lemma}
%
%\usecolortheme{orchid}
\setbeamertemplate{theorems}[numbered]
%\usecolortheme{orchid}
%\setbeamertemplate{theorems}[ams style]
%\setbeamertemplate{theorems}[numbered]

%% Listings
\usepackage{listings}
\definecolor{mygreen}{rgb}{0,0.6,0}
\definecolor{mygray}{rgb}{0.5,0.5,0.5}
\definecolor{mymauve}{rgb}{0.58,0,0.82}
\definecolor{mygold}{rgb}{1,0.843,0}
\definecolor{myblue}{rgb}{0.537,0.812,0.941}

\definecolor{lgreen}{rgb}{0.6,0.9,.6}
\definecolor{lred}{rgb}{1,0.5,.5}

\lstloadlanguages{R}
\lstset{ %
  language=R,
  backgroundcolor=\color{black!95},   % choose the background color
  basicstyle=\footnotesize\ttfamily,        % size of fonts used for the code
  breaklines=true,                 % automatic line breaking only at whitespace
  captionpos=b,                    % sets the caption-position to bottom
  commentstyle=\color{mygreen},    % comment style
  escapeinside={\%*}{*)},          % if you want to add LaTeX within your code
  keywordstyle=\color{myblue},       % keyword style
  stringstyle=\color{mygold},     % string literal style
  keepspaces=true,
  columns=fullflexible,
  tabsize=4,
}
% Could also do (in lstset)
% basicstyle==\fontfamily{pcr}\footnotesize


% Get rid of navigation stuff
\setbeamertemplate{navigation symbols}{}

% Set footline/header line
\setbeamertemplate{footline}
{%
\quad p. \insertpagenumber \quad--\quad \insertsection\vskip2pt
}
% \setbeamertemplate{headline}
% {%
% \quad\insertsection\hfill p. \insertpagenumber\quad\mbox{}\vskip2pt
% }


\makeatletter
\newlength\beamerleftmargin
\setlength\beamerleftmargin{\Gm@lmargin}
\makeatother

%%%%%%% 
%% Definitions in yellow boxes
\usepackage{etoolbox}
\setbeamercolor{block title}{use=structure,fg=structure.fg,bg=structure.fg!05!bg}
\setbeamercolor{block body}{parent=normal text,use=block title,bg=block title.bg!20!bg}

\BeforeBeginEnvironment{definition}{%
	\setbeamercolor{block title}{fg=black,bg=yellow!20!white}
	\setbeamercolor{block body}{fg=black, bg=yellow!05!white}
}
\AfterEndEnvironment{definition}{
	\setbeamercolor{block title}{use=structure,fg=structure.fg,bg=structure.fg!20!bg}
	\setbeamercolor{block body}{parent=normal text,use=block title,bg=block title.bg!50!bg, fg=black}
}
\BeforeBeginEnvironment{importanttheorem}{%
	\setbeamercolor{block title}{fg=black,bg=red!20!white}
	\setbeamercolor{block body}{fg=black, bg=red!05!white}
}
\AfterEndEnvironment{importanttheorem}{
	\setbeamercolor{block title}{use=structure,fg=structure.fg,bg=structure.fg!20!bg}
	\setbeamercolor{block body}{parent=normal text,use=block title,bg=block title.bg!50!bg, fg=black}
}
\BeforeBeginEnvironment{theorem}{%
	\setbeamercolor{block title}{fg=white,bg=red!30!black}
	\setbeamercolor{block body}{fg=white, bg=red!10!black}
}
\AfterEndEnvironment{theorem}{
	\setbeamercolor{block title}{use=structure,fg=structure.fg,bg=structure.fg!20!bg}
	\setbeamercolor{block body}{parent=normal text,use=block title,bg=block title.bg!50!bg, fg=black}
}
\BeforeBeginEnvironment{importantproperty}{%
	\setbeamercolor{block title}{fg=black,bg=red!50!white}
	\setbeamercolor{block body}{fg=black, bg=red!30!white}
}
\AfterEndEnvironment{importantproperty}{
	\setbeamercolor{block title}{use=structure,fg=structure.fg,bg=structure.fg!20!bg}
	\setbeamercolor{block body}{parent=normal text,use=block title,bg=block title.bg!50!bg, fg=black}
}


%%%%%%%%%%%%%%%%%
\usepackage{tikz}
\usetikzlibrary{shapes,arrows}
\usetikzlibrary{positioning}
\usetikzlibrary{shapes.symbols,shapes.callouts,patterns}
\usetikzlibrary{calc,fit}
\usetikzlibrary{backgrounds}
\usetikzlibrary{decorations.pathmorphing,fit,petri}
\usetikzlibrary{automata}
\usetikzlibrary{fadings}
\usetikzlibrary{patterns,hobby}

\usepackage{pgfplots}
\pgfplotsset{compat=1.6}
\pgfplotsset{ticks=none}

\usetikzlibrary{decorations.markings}
\usetikzlibrary{arrows.meta}
\tikzset{>=stealth}

\tikzstyle{cloud} = [draw, 
ellipse,
fill=red!20, 
node distance=0.87cm,
minimum height=2em]
\tikzstyle{line} = [draw, 
-latex', 
color=yellow]


% Beginning of a section
% \AtBeginSection[]{
% 	{
% 		\setbeamercolor{background canvas}{bg=orange!10}
% 		\begin{frame}[noframenumbering,plain]
% 			\framesubtitle{\nameofthepart Chapter \insertromanpartnumber \ -- \iteminsert{\insertpart}}
% 			\tableofcontents[currentsection,currentsubsection]
% 		\end{frame}
% 	\addtocounter{page}{-1}
% 	%\addtocounter{framenumber}{-1} 
% 	}
% }


%%% SLIDES COLOURING

%\usecolortheme{owl}

\setbeamerfont{frametitle}{series=\bfseries}
\setbeamercolor{frametitle}{fg=black!05,bg=black}

\setbeamerfont{framesubtitle}{size=\normalfont\tiny}
\setbeamercolor{framesubtitle}{fg=black!05}

\setbeamercolor{background canvas}{bg=black}
\setbeamercolor{normal text}{fg=black!10}



\definecolor{bottomcolour}{rgb}{0.32,0.3,0.38}
\definecolor{middlecolour}{rgb}{0.08,0.08,0.16}
\definecolor{mycolor}{rgb}{0.4,0.4, 0.4}
% Beginning of a section
\AtBeginSection[]{
	{
		%\setbeamercolor{background canvas}[vertical shading][top=bottomcolour, middle=middlecolour, bottom=black]
		\setbeamertemplate{background canvas}[vertical shading][bottom=bottomcolour,top=black!20]
    % \setbeamertemplate{background canvas}{
    %   \begin{tikzpicture}%[remember picture,overlay]
    %     \shade[top color=yellow!75!green!33,
    %     bottom color=blue!66!green!33,
    %     middle color=blue!6!green!33]
    %   \end{tikzpicture}
    % }
    \begin{frame}[noframenumbering,plain]
			\framesubtitle{\nameofthepart Chapter \insertromanpartnumber \ -- \iteminsert{\insertpart}}
			\tableofcontents[currentsection,currentsubsection]
		\end{frame}
	\addtocounter{page}{-1}
	}
}
% Beginning of a section
\AtBeginSubsection[]{
	{
		%\setbeamercolor{background canvas}[vertical shading][top=bottomcolour, middle=middlecolour, bottom=black]
		%\setbeamertemplate{background canvas}[vertical shading][bottom=bottomcolour,top=black!20]
    \setbeamertemplate{background canvas}{
      \begin{tikzpicture}%[remember picture,overlay]
        \shade[top color=yellow!75!green!33,
        bottom color=blue!66!green!33,
        middle color=blue!6!green!33]
      \end{tikzpicture}
    }
    \begin{frame}[noframenumbering,plain]
			\framesubtitle{\nameofthepart Chapter \insertromanpartnumber \ -- \iteminsert{\insertpart}}
			\tableofcontents[currentsection,currentsubsection]
		\end{frame}
	\addtocounter{page}{-1}
	}
}

% Colours for special pages
\def\extraContent{yellow!20}

%% Allow to change slide colour
%% From: https://tex.stackexchange.com/questions/8043/change-the-background-color-of-a-frame-in-beamer
\defbeamertemplate*{background canvas}{mydefault}{%
  \ifbeamercolorempty[bg]{background canvas}{}{\color{bg}\vrule width\paperwidth height\paperheight}% copied beamer default here
}
\defbeamertemplate*{background canvas}{bg}{%
  \color{lightgray!20}\vrule width\paperwidth height\paperheight% added bg color
}
\BeforeBeginEnvironment{frame}{%
  \setbeamertemplate{background canvas}[mydefault]%
}
\makeatletter
\define@key{beamerframe}{bg}[true]{%
  \setbeamertemplate{background canvas}[bg]%
}
\makeatother
% Use with
%\begin{frame}
% \frametitle{Normal}
%\end{frame} 
%\begin{frame}[bg]
% \frametitle{With bg}
%\end{frame}


%% Vertical alignment on pages
%% From: https://tex.stackexchange.com/questions/148365/how-do-i-ask-beamer-to-exactly-fill-up-a-slide
%% Turn on with
%% \stretchon
%% (outside slide), and off with
%% \stretchoff
% \def\itemsymbol{$\blacktriangleright$}
% \let\svpar\par
% \let\svitemize\itemize
% \let\svenditemize\enditemize
% \let\svitem\item
% \let\svcenter\center
% \let\svendcenter\endcenter
% \let\svcolumn\column
% \let\svendcolumn\endcolumn
% \def\newitem{\renewcommand\item[1][\itemsymbol]{\vfill\svitem[##1]}}%
% \def\newpar{\def\par{\svpar\vfill}}%
% \newcommand\stretchon{%
%   \newpar%
%   \renewcommand\item[1][\itemsymbol]{\svitem[##1]\newitem}%
%   \renewenvironment{itemize}%
%     {\svitemize}{\svenditemize\newpar\par}%
%   \renewenvironment{center}%
%     {\svcenter\newpar}{\svendcenter\newpar}%
%   \renewenvironment{column}[2]%
%     {\svcolumn{##1}\setlength{\parskip}{\columnskip}##2}%
%     {\svendcolumn\vspace{\columnskip}}%
% }
% \newcommand\stretchoff{%
%   \let\par\svpar%
%   \let\item\svitem%
%   \let\itemize\svitemize%
%   \let\enditemize\svenditemize%
%   \let\center\svcenter%
%   \let\endcenter\svendcenter%
%   \let\column\svcolumn%
%   \let\endcolumn\svendcolumn%
% }
% \newlength\columnskip
% \columnskip 0pt



\title{Environmentally Transmitted Pathogens}
\subtitle{Models}
\author{Julien Arino}
\date{January 2023}


\begin{document}
%\stretchon

% The title page
\begin{frame}[noframenumbering,plain]
  \titlepage
\end{frame}
\addtocounter{page}{-1}

%%%%%%%%%%%%%%%%%%%%
%%%%%%%%%%%%%%%%%%%%
%%%%%%%%%%%%%%%%%%%%
%%%%%%%%%%%%%%%%%%%%
\section{A few models of Cvjetanovi\'c}

\begin{frame}
  \begin{itemize}
    \item Cvjetanovi\'c, Grab, Uemura \& World Health Organization. \href{https://apps.who.int/iris/handle/10665/43033}{Dynamics of acute bacterial diseases : epidemiological models and their application in public health}. World Health Organization (1978)  
    \item Briscoe. \href{https://doi.org/10.1093/ije/9.3.265}{On the use of simple analytic mathematical models of communicable diseases}.
    \emph{International Journal of Epidemiology} \textbf{9}(3) (1980)
  \end{itemize}
  \vfill
  Models of (Branko) Cvjetanovi\'c are in discrete time and quite detailed on the epi side
\end{frame}
%%%%%%%%%%%%%%%%%%%%
%%%%%%%%%%%%%%%%%%%%
\subsection{A tetanus model}

\begin{frame}{Tetanus}
  Bacterial infection caused by \emph{Clostridium tetani}
  \vfill
  Spores everywhere in the environment, including soil, dust, and manure
  \vfill
  Spores develop into bacteria when they enter the body
  \vfill
  Not spread P2P (so not a classic model)
  \vfill
  Incubation period usually 3-21 days (average 8 days). Can range from 1 day to several months, depending on the kind of wound. Most cases occur within 14 days
  \vfill
  1 to 2 in 10 cases are fatal
\end{frame}

\maxFrameImage{../FIGS/tetanus-vacc-media_inverted.jpg}

\begin{frame}
  \begin{center}
    \def\vertskip{*-2}
    \def\horzskip{*2}
    \begin{tikzpicture}[scale=0.75, transform shape]
      \node [rectangle, fill=gray!10, text=black] at (-2\horzskip,0\vertskip) (S_nb) {Newborn};
      \node [rectangle, fill=gray!10, text=black] at (2\horzskip,0\vertskip) (S) {Susceptible population};
      \node [rectangle, fill=gray!10, text=black] at (-2\horzskip,1\vertskip) (L_nb) {Incubating newborn};
      \node [rectangle, fill=gray!10, text=black] at (2\horzskip,1\vertskip) (L) {Incubating population};
      \node [rectangle, fill=gray!10, text=black] at (-2\horzskip,2\vertskip) (I_nb) {Sick newborn};
      \node [rectangle, fill=gray!10, text=black] at (2\horzskip,2\vertskip) (I) {Sick population};
      \node [rectangle, fill=gray!10, text=black] at (0\horzskip,3\vertskip) (D) {Tetanus deaths};
      \node [rectangle, fill=gray!10, text=black] at (0\horzskip,4\vertskip) (R) {Active immunity 10 years};
      \node [rectangle, fill=gray!10, text=black] at (0\horzskip,5\vertskip) (R_p) {Passive immunity 6 months};
      %% Flows
      \path [line, very thick] (S_nb) to node [midway, above] (TextNode) {} (S);
      \path [line, very thick] (S_nb) to node [midway, above] (TextNode) {} (L_nb);
      \path [line, very thick] (L_nb) to node [midway, above] (TextNode) {} (I_nb);
      \path [line, very thick] (I_nb) to node [midway, above] (TextNode) {} (D);
      \path [line, very thick] (I_nb) to node [midway, above] (TextNode) {} (R);
      \path [line, very thick, bend right=60] (S_nb.west) to node [midway, above] (TextNode) {} (R_p.west);
      \path [line, very thick] (I_nb) to node [midway, above] (TextNode) {} (S.-170);
      \path [line, very thick] (I_nb) to node [midway, above] (TextNode) {} (R);
      \path [line, very thick] (I_nb) to node [midway, above] (TextNode) {} (D);
      \path [line, very thick] (S) to node [midway, above] (TextNode) {} (L);
      \path [line, very thick] (L) to node [midway, above] (TextNode) {} (I);
      \path [line, very thick, bend left=75] (I) to node [midway, above] (TextNode) {} (S);
      \path [line, very thick] (I) to node [midway, above] (TextNode) {} (R);
      \path [line, very thick] (I) to node [midway, above] (TextNode) {} (D);
      \path [line, very thick, bend right=60] (R.east) to node [midway, above] (TextNode) {} (S.-10);      
      \path [line, very thick, bend right=60, anchor=east] (R_p.east) to node [midway, above] (TextNode) {} (S.east);      
    \end{tikzpicture}    
  \end{center}  
\end{frame}

\begin{frame}
  \begin{center}
    \def\vertskip{*-1.5}
    \def\horzskip{*1}
    \begin{tikzpicture}[scale=1, transform shape]
      \node [rectangle, fill=gray!10, text=black] at (-2\horzskip,0\vertskip) (S_nb) {$S_b$};
      \node [rectangle, fill=gray!10, text=black] at (2\horzskip,0\vertskip) (S) {$S$};
      \node [rectangle, fill=gray!10, text=black] at (-2\horzskip,1\vertskip) (L_nb) {$L_b$};
      \node [rectangle, fill=gray!10, text=black] at (2\horzskip,1\vertskip) (L) {$L$};
      \node [rectangle, fill=gray!10, text=black] at (-2\horzskip,2\vertskip) (I_nb) {$I_b$};
      \node [rectangle, fill=gray!10, text=black] at (2\horzskip,2\vertskip) (I) {$I$};
      \node [rectangle, fill=gray!10, text=black] at (0\horzskip,3\vertskip) (D) {$D$};
      \node [rectangle, fill=gray!10, text=black] at (0\horzskip,4\vertskip) (R) {$R$};
      \node [rectangle, fill=gray!10, text=black] at (0\horzskip,5\vertskip) (R_p) {$R_b$};
      %% Flows
      \path [line, very thick] (S_nb) to node [midway, above] (TextNode) {} (S);
      \path [line, very thick] (S_nb) to node [midway, above] (TextNode) {} (L_nb);
      \path [line, very thick] (L_nb) to node [midway, above] (TextNode) {} (I_nb);
      \path [line, very thick] (I_nb) to node [midway, above] (TextNode) {} (D);
      \path [line, very thick] (I_nb) to node [midway, above] (TextNode) {} (R);
      \path [line, very thick, bend right=60] (S_nb.west) to node [midway, above] (TextNode) {} (R_p.west);
      \path [line, very thick] (I_nb) to node [midway, above] (TextNode) {} (S.-170);
      \path [line, very thick] (I_nb) to node [midway, above] (TextNode) {} (R);
      \path [line, very thick] (I_nb) to node [midway, above] (TextNode) {} (D);
      \path [line, very thick] (S) to node [midway, above] (TextNode) {} (L);
      \path [line, very thick] (L) to node [midway, above] (TextNode) {} (I);
      \path [line, very thick, bend left=55] (I) to node [midway, above] (TextNode) {} (S.south west);
      \path [line, very thick] (I) to node [midway, above] (TextNode) {} (R);
      \path [line, very thick] (I) to node [midway, above] (TextNode) {} (D);
      \path [line, very thick, bend right=60] (R.east) to node [midway, above] (TextNode) {} (S.-10);      
      \path [line, very thick, bend right=60, anchor=east] (R_p.east) to node [midway, above] (TextNode) {} (S.east);      
    \end{tikzpicture}    
  \end{center}  
\end{frame}


%%%%%%%%%%%%%%%%%%%%
%%%%%%%%%%%%%%%%%%%%
\subsection{A cholera model}

\begin{frame}{A cholera model}
  \begin{center}
    \def\vertskip{*1.75}
    \begin{tikzpicture}[scale=0.75, transform shape]
      \node [rectangle, fill=gray!10, text=black] at (0,0\vertskip) (S_nb) {Newborn};
      \node [rectangle, fill=gray!10, text=black] at (0,1\vertskip) (S) {Susceptible};
      \node [rectangle, fill=gray!10, text=black] at (4,1\vertskip) (V) {Vaccinated (immune)};
      \node [rectangle, fill=gray!10, text=black, text width=2.5cm] at (-4,2\vertskip) (C_c) {Contact carrier (5-7 days)};
      \node [rectangle, fill=gray!10, text=black, text width=2.5cm] at (0,2\vertskip) (L_I) {Incubating infectious (1-3 days)};
      \node [rectangle, fill=gray!10, text=black, text width=2.5cm] at (4,2\vertskip) (L_nI) {Incubating non infectious (1-3 days)};
      \node [rectangle, fill=gray!10, text=black] at (0,3\vertskip) (I_I) {Sick infectious (3-6 days)};
      \node [rectangle, fill=gray!10, text=black, text width=2.5cm] at (4,4\vertskip) (C_r) {Convalescent carrier (14-21 days)};
      \node [rectangle, fill=gray!10, text=black, text width=2.5cm] at (0,4\vertskip) (R_s) {Healthy resistant};
      \node [rectangle, fill=gray!10, text=black, text width=2.5cm] at (-4,4\vertskip) (D) {Deaths from cholera};
      \node [rectangle, fill=gray!10, text=black] at (4,5\vertskip) (C_p) {Chronic carrier};
      %% Flows
      \path [line, very thick] (S_nb) to node [midway, above] (TextNode) {} (S);
      \path [line, very thick] (S.north east) to node [midway, above] (TextNode) {} (V.north west);
      \path [line, very thick] (V.south west) to node [midway, above] (TextNode) {} (S.south east);
      \path [line, very thick] (S) to node [midway, above] (TextNode) {} (C_c);
      \path [line, very thick] (S) to node [midway, above] (TextNode) {} (L_I);
      \path [line, very thick] (S) to node [midway, above] (TextNode) {} (L_nI);
      \path [line, very thick] (L_I) to node [midway, above] (TextNode) {} (I_I);
      \path [line, very thick] (L_nI) to node [midway, above] (TextNode) {} (I_I);
      \path [line, rounded corners, very thick] (C_c) -- (-4,0.5\vertskip) -- (-1.5,0.5\vertskip) -- (S.south);
      \path [line, very thick] (I_I) to node [midway, above] (TextNode) {} (D);
      \path [line, very thick] (I_I) to node [midway, above] (TextNode) {} (R_s);
      \path [line, very thick] (I_I) to node [midway, above] (TextNode) {} (C_r);
      \path [line, very thick] (C_r) to node [midway, above] (TextNode) {} (C_p);
      \path [line, rounded corners, very thick] (I_I.east) -- (6,3\vertskip) -- (6,0.5\vertskip) -- (1.5,0.5\vertskip) -- (S.south);
      \path [line, rounded corners, very thick] (C_r.east) -- (6,4\vertskip) -- (6,0.5\vertskip) -- (1.5,0.5\vertskip) -- (S.south);
      \path [line, rounded corners, very thick] (C_c) -- (-6,2\vertskip) -- (-6,5\vertskip) -- (C_p);
      \path [line, rounded corners, very thick] (C_c) -- (-6,2\vertskip) -- (-6,5\vertskip) -- (-1.5,5\vertskip) -- (R_s);
      \path [line, very thick] (C_r) to node [midway, above] (TextNode) {} (R_s);
      \path [line, rounded corners, very thick] (R_s) -- (0,4.6\vertskip) -- (7,4.6\vertskip) -- (7,0.5\vertskip) -- (1.5,0.5\vertskip) -- (S.south);
    \end{tikzpicture}    
  \end{center}  
\end{frame}

%%%%%%%%%%%%%%%%%%%%
%%%%%%%%%%%%%%%%%%%%
%%%%%%%%%%%%%%%%%%%%
%%%%%%%%%%%%%%%%%%%%
\section{A model of Capasso for ETP}

\begin{frame}{A minimal model of V. Capasso}
  \begin{center}
    \def\vertskip{*1.75}
    \def\horzskip{*2}
    \begin{tikzpicture}[scale=1.5, transform shape]
      \node [circle, fill=gray!10, text=black] at (-1\horzskip,0\vertskip) (H) {$H$};
      \node [circle, fill=gray!10, text=black] at (1\horzskip,0\vertskip) (E) {$E$};
      %% Flows between
      \path [line, very thick, bend left] (H) to node [near end, above] (TextNode) {$c_HH$} (E);
      \path [line, very thick, bend left] (E) to node [near end, below] (TextNode) {$g(E)$} (H);
      %% Flows out of
      \path [line, very thick] (H) to node [midway, above] (TextNode) {$\gamma_HH$} (-1.75\horzskip,0\vertskip);
      \path [line, very thick] (E) to node [midway, below] (TextNode) {$d_EE$} (1.75\horzskip,0\vertskip);
      %% Vertical line
      \draw [dashed, very thick, color=red] (0\horzskip,-1\vertskip) -- (0\horzskip,1\vertskip);
      %% Text
      \node [color=red] at (-1\horzskip,-1\vertskip) {Human population};
      \node [color=red] at (1\horzskip,-1\vertskip) {Environment};
    \end{tikzpicture}    
  \end{center}  
  $1/\gamma_H$ mean infectious period, $1/d_E$ mean lifetime of the agent in the environment, $c_H$ growth rate of the agent due to the human population, $g(E)$ ``force of infection'' (I would say ``incidence'') of the agent on human population
\end{frame}

\begin{frame}{Incidence function}
  \begin{equation}
    \label{eq:incidence_function_Capasso}
    g(E) = N\beta ph(E)
  \end{equation}
  where
  \begin{itemize}
    \item $N$ total human population
    \item $\beta$ fraction of susceptible individuals in $N$
    \item $p$ fraction exposed to contaminated environment per unit time (``probability per unit time to have a ``snack'' of contaminated food'')
    \item $h(E)$ probability for an exposed susceptible to get the infection
  \end{itemize}
  Typically, we would assume $p$ and $\beta$ independent of $E$ and $H$ and $h$ to be saturating
\end{frame}

\begin{frame}
  To ensure \eqref{eq:incidence_function_Capasso} satisfies these conditions, we can assume
  \begin{itemize}
    \item $0<g(e_1)<g(e_2)$ for $0<e_1<e_2$
    \item $g(0)=0$
    \item $g''(z)<0$ for all $z>0$
    \item $0<g'_+(0)<\infty$  (right derivative)
    \item $\lim_{z\to\infty}\frac{g(z)}{z}<\frac{d_E\gamma_H}{c_H}$
  \end{itemize}
  \vfill
  Of course, we also assume $d_E,c_H,\gamma_H>0$
\end{frame}

\begin{frame}{The model}
  \begin{subequations}
    \label{sys:capasso_EH}
    \begin{align}
      E' &= c_HH-d_EE \label{sys:capasso_EH_dE} \\
      H' &= g(E)-\gamma_HH \label{sys:capasso_EH_dH}
    \end{align}
  \end{subequations}
  \begin{center}
    \def\vertskip{*1.75}
    \def\horzskip{*2}
    \begin{tikzpicture}[scale=1, transform shape]
      \node [circle, fill=gray!10, text=black] at (-1\horzskip,0\vertskip) (H) {$H$};
      \node [circle, fill=gray!10, text=black] at (1\horzskip,0\vertskip) (E) {$E$};
      %% Flows between
      \path [line, very thick, bend left] (H) to node [near end, above] (TextNode) {$c_HH$} (E);
      \path [line, very thick, bend left] (E) to node [near end, below] (TextNode) {$g(E)$} (H);
      %% Flows out of
      \path [line, very thick] (H) to node [midway, above] (TextNode) {$\gamma_HH$} (-1.75\horzskip,0\vertskip);
      \path [line, very thick] (E) to node [midway, below] (TextNode) {$d_EE$} (1.75\horzskip,0\vertskip);
      %% Vertical line
      \draw [dashed, very thick, color=red] (0\horzskip,-1\vertskip) -- (0\horzskip,1\vertskip);
      %% Text
      \node [color=red] at (-1\horzskip,-1\vertskip) {Human population};
      \node [color=red] at (1\horzskip,-1\vertskip) {Environment};
    \end{tikzpicture}    
  \end{center}
  \vfill
  Pay attention to the flows..! $E'$ does not have a $-g(E)$ and $H'$ does not have $-c_HH$. Why?
\end{frame}

\begin{frame}
  Let
  \begin{equation}
    \label{eq:R0_capasso}
    \R_0 = \frac{g'_+(0)c_H}{d_E\gamma_H}
  \end{equation}
  \begin{theorem}
    \begin{itemize}
      \item If $0<\R_0<1$, then \eqref{sys:capasso_EH} admits only the trivial equilibrium in the positive orthant, which is GAS
      \item If $\R_0>1$, then two EP exist: $(0,0)$, which is unstable, and $z^\star=(E^\star,H^\star)$ with $E^\star,H^\star>0$, GAS in $\IR_+^2\setminus\{0,0\}$
    \end{itemize}
  \end{theorem}
\end{frame}


\begin{frame}{Adding a periodic component}
  Assume $p$ in \eqref{eq:incidence_function_Capasso} takes the form 
  \begin{equation}
    p(t)=p(t+\omega)>0,\quad t\in\IR
  \end{equation}
  i.e., $p$ has period $\omega$. So we now consider the incidence
  \begin{equation}
    \label{eq:incidence_Capasso_periodic}
    g(t,E)=p(t)h(E)
  \end{equation}
  with $h$ having the properties prescribed earlier.
  Letting 
  \begin{equation}
    p_{min} := \min_{0\leq t\leq\omega}p(t),\quad
    p_{max} := \max_{0\leq t\leq\omega}p(t)
  \end{equation}
  then we require that 
  \begin{equation}
    \lim_{z\to\infty}\frac{g(z)}{z}<\frac{d_E\gamma_H}{c_Hp_{max}}
  \end{equation}
\end{frame}


\begin{frame}
  Let
  \begin{equation}
    \label{eq:R0_capasso_periodic}
    \R_0^{min} = \frac{c_Hp_{min}h'_+(0)}{d_E\gamma_H},\quad 
    \R_0^{max} = \frac{c_Hp_{max}h'_+(0)}{d_E\gamma_H}
  \end{equation}
  \begin{theorem}
    \begin{itemize}
      \item If $0<\R_0^{max}<1$, then \eqref{sys:capasso_EH} with incidence \eqref{eq:incidence_Capasso_periodic} always goes to extinction
      \item If $\R_0^{min}>1$, then a unique nontrivial periodic endemic state exists for \eqref{sys:capasso_EH} with incidence \eqref{eq:incidence_Capasso_periodic}
    \end{itemize}
  \end{theorem}
\end{frame}



%%%%%%%%%%%%%%%%%%%%
%%%%%%%%%%%%%%%%%%%%
%%%%%%%%%%%%%%%%%%%%
%%%%%%%%%%%%%%%%%%%%
\section{A model for zoonotic transmission of waterborne disease}

\begin{frame}{Zoonotic transmission of waterborne disease}
  Waters, Hamilton, Sidhu, Sidhu, Dunbar. \href{https://doi-org.uml.idm.oclc.org/10.1007/s11538-015-0136-y}{Zoonotic transmission of waterborne disease: a mathematical model}. 
  \emph{Bull Math Biol}  (2016)

  Used for instance to model Giardia transmission from possums to humans
\end{frame}

\begin{frame}
  \begin{center}
    \def\vertskip{*2}
    \def\horzskip{*3}
    \begin{tikzpicture}[scale=1, transform shape]
      \node [rectangle, fill=gray!10, text=black] at (-1\horzskip,0\vertskip) (S_H) {Susceptible humans};
      \node [rectangle, fill=gray!10, text=black] at (1\horzskip,0\vertskip) (I_H) {Infectious humans};
      \node [rectangle, fill=gray!10, text=black] at (-1\horzskip,2\vertskip) (S_A) {Susceptible animals};
      \node [rectangle, fill=gray!10, text=black] at (1\horzskip,2\vertskip) (I_A) {Infectious animals};
      \node [rectangle, fill=gray!10, text=black] at (0\horzskip,1\vertskip) (W) {Live oo/cysts in water};
      %% Flows 
      \path [line, very thick] (S_H.south east) to node [midway, below] (TextNode) {P2P transmission} (I_H.south west);
      \path [line, dashed, very thick] (S_H.north east) to node [near end, above] (TextNode) {conversion of oo/cysts to infection} (I_H.north west);
      \path [line, very thick, bend left] (I_H) to node [midway, below] (TextNode) {recovery} (S_H);
      \path [line, very thick] (S_A.north east) to node [midway, above] (TextNode) {A2A transmission} (I_A.north west);
      \path [line, very thick] (I_A.south west) to node [midway, below] (TextNode) {recovery} (S_A.south east);
      %% Flows of W
      \path [line, dashed, very thick] (W) to node [midway, above] (TextNode) {Death of oo/cysts in water} (-2\horzskip,1\vertskip);
      \path [line, dashed, very thick] (W) to node [near start, left] (TextNode) {pick up rate} (0\horzskip,0.35\vertskip);
      \path [line, dashed, very thick] (I_A) to node [midway, below] (TextNode) {deposit rate} (W);
    \end{tikzpicture}    
  \end{center}  
\end{frame}

\begin{frame}
  \begin{center}
    \def\vertskip{*2}
    \def\horzskip{*2}
    \begin{tikzpicture}[scale=1.25, transform shape]
      \node [circle, fill=gray!10, text=black] at (-1\horzskip,0\vertskip) (S_H) {$S_H$};
      \node [circle, fill=gray!10, text=black] at (1\horzskip,0\vertskip) (I_H) {$I_H$};
      \node [circle, fill=gray!10, text=black] at (-1\horzskip,2\vertskip) (S_A) {$S_A$};
      \node [circle, fill=gray!10, text=black] at (1\horzskip,2\vertskip) (I_A) {$I_A$};
      \node [circle, fill=gray!10, text=black] at (0\horzskip,1\vertskip) (W) {$W$};
      %% Flows between
      \path [line, very thick] (S_H) to node [midway, below] (TextNode) {$\beta_H$} (I_H);
      \path [line, dashed, very thick, bend left] (S_H) to node [near end, above] (TextNode) {$\rho$} (I_H);
      \path [line, very thick, bend left] (I_H) to node [midway, below] (TextNode) {$\gamma_HI_H$} (S_H);
      \path [line, very thick, bend left] (S_A) to node [midway, above] (TextNode) {$\beta_A$} (I_A);
      \path [line, very thick, bend left] (I_A) to node [midway, above] (TextNode) {$\gamma_AI_A$} (S_A);
      %% Flows of W
      \path [line, dashed, very thick] (W) to node [midway, above] (TextNode) {$\mu W$} (-0.75\horzskip,1\vertskip);
      \path [line, dashed, very thick] (W) to node [near start, left] (TextNode) {$\eta$} (0\horzskip,0.35\vertskip);
      \path [line, dashed, very thick] (I_A) to node [midway, below, sloped] (TextNode) {$\alpha I_A$} (W);
    \end{tikzpicture}    
  \end{center}  
\end{frame}


\begin{frame}{The full model}
  \begin{subequations}
    \label{sys:WaterHamilton_etal}
    \begin{align}
      S_A' &= -\beta_AS_AI_A+\gamma_AI_A \\
      I_A' &= \beta_AS_AI_A-\gamma_AI_A \\
      W' &= \alpha I_A-\eta W(S_H+I_H)-\mu W \\
      S_H' &= -\rho\eta WS_H-\beta_HS_HI_H+\gamma_HI_H \\
      I_H' &= \rho\eta WS_H+\beta_HS_HI_H-\gamma_HI_H 
    \end{align}
  \end{subequations}
  \vfill
  Considered with $N_A=S_A+I_A$ and $N_H=S_H+I_H$ constant
\end{frame}

\begin{frame}{Simplified model}
  Because $N_A$ and $N_H$ are constant, \eqref{sys:WaterHamilton_etal} can be simplified:
  \begin{subequations}
    \label{sys:WaterHamilton_etal_simplified}
    \begin{align}
      I_A' &= \beta_AN_AI_A-\gamma_AI_A-\beta_AI_A^2 \\
      W' &= \alpha I_A-\eta WN_H-\mu W \\
      I_H' &= \rho\eta W(N_H-I_H)+\beta_HN_HI_H-\gamma_HI_H-\beta_HI_H^2 
    \end{align}
  \end{subequations}
  \vfill
  Three EP: DFE $(0,0,0)$; endemic disease in humans because of H2H transmission; endemic in both H and A because of W
\end{frame}


\begin{frame}
  Three EP: DFE $(0,0,0)$; endemic disease in humans because of H2H transmission; endemic in both H and A because of W
  \vfill
  Let
  \begin{equation}
    \R_{0A} = \frac{\beta_A}{\gamma_A}N_A\quad\text{and}\quad
    \R_{0H} = \frac{\beta_H}{\gamma_H}N_H
  \end{equation}
  \vfill
  \begin{itemize}
    \item DFE LAS if $\R_{0A}<1$ and $\R_{0H}<1$, unstable if $\R_{0A}>1$ or $\R_{0H}>1$
    \item If $\R_{0H}>1$ and $\R_{0A}<1$, \eqref{sys:WaterHamilton_etal_simplified} goes to EP with endemicity only in humans
    \item Endemic EP with both A and H requires $\R_{0A}>1$ and $\R_{0H}<1$
  \end{itemize}
  Note that proof is \textbf{not} global
\end{frame}


%%%%%%%%%%%%%%%%%%%%
%%%%%%%%%%%%%%%%%%%%
%%%%%%%%%%%%%%%%%%%%
%%%%%%%%%%%%%%%%%%%%
\section{A few models of schistosomiasis}

%%%%%%%%%%%%%%%%%%%%
%%%%%%%%%%%%%%%%%%%%
\subsection{A first model of Woolhouse}
\begin{frame}{A model of Woolhouse}
  Woolhouse. \href{}{On the application of mathematical models of schistosome transmission dynamics. I. Natural transmission}. \emph{Acta Tropica} \textbf{49}:241-270 (1991)
\end{frame}

\begin{frame}{The model}
  Population of $H$ individuals using a body of water containing $N$ snails
  \vfill
  $i_H$ mean number of schistosomes per person and $i_S$ the proportion of patent infections in snails 
  (prevalence)
  \vfill
  \begin{subequations}
    \label{sys:Woolhouse}
    \begin{align}
      i_H' &= \alpha Ni_S-\gamma i_H \\
      i_S' &= \beta Hi_H(1-i_S)-\mu_2 i_S
    \end{align}
  \end{subequations}
  \begin{itemize}
    \item $\alpha$ number of schistosomes produced per person per infected snail per unit time
    \item $1/\gamma$ average life expectancy of a schistosome
    \item $1/\mu_2$ average life expectancy of an infected snail
    \item $\beta$ transmission parameter
  \end{itemize}
\end{frame}

\begin{frame}
  Let the basic reproductive rate for schistosomes be
  \begin{equation}
    \label{eq:R0_Woolhouse}
    \R_0 = \frac{\alpha N\beta H}{\gamma\mu_2}
  \end{equation}
  \vfill
  \eqref{sys:Woolhouse} has two EP
  \begin{itemize}
    \item $(i_H^\star,i_S^\star)=(0,0)$, LAS when $\R_0<1$ and unstable when $\R_0>1$
    \item $(i_H^\star,i_S^\star)=\left(\dfrac{\alpha N}{\gamma}-\dfrac{\mu_2}{\beta H},1-\dfrac{1}{\R_0}\right)$, which only ``exists'' when $\R_0>1$ (and is LAS then)
  \end{itemize}
\end{frame}

%%%%%%%%%%%%%%%%%%%%
%%%%%%%%%%%%%%%%%%%%
\subsection{A second model of Woolhouse -- Latency}
\begin{frame}{Extending the model}
  Interval between infection of a snail and onset of patency (release of cercariae) is \emph{prepatent} or \emph{latent} period
  \begin{subequations}
    \label{sys:Woolhouse2}
    \begin{align}
      i_H' &= \alpha Ni_S-\gamma i_H \\
      \ell_S' &= \beta Hi_H(1-\ell_S-i_S)-\sigma\ell_S-\mu_1\ell_S \\
      i_S' &= \sigma\ell_S-\mu_2 i_S
    \end{align}
  \end{subequations}
  \begin{itemize}
    \item $1/\sigma$ average duration of prepatent period
    \item $f=\sigma/(\sigma+\mu_1)$ fraction of infected snails surviving prepatent period
  \end{itemize}
\end{frame}

\begin{frame}
  The basic reproductive rate for schistosomes is now
  \begin{equation}
    \label{eq:R0_Woolhouse2}
    \R_0 = f\frac{\alpha N\beta H}{\gamma\mu_2}
  \end{equation}
  \vfill
  \eqref{sys:Woolhouse2} has endemic EP
  \[
    (i_H^\star,i_S^\star)=\left(\dfrac{\alpha N\sigma}{\gamma(\sigma+\mu_2)}-\dfrac{\mu_2(\sigma+\mu_1)}{\beta H(\sigma+\mu_2)},\frac{\sigma}{\sigma+\mu_2}\left(1-\dfrac{1}{\R_0}\right)\right)
  \]
\end{frame}

\begin{frame}
  Also has models
  \begin{itemize}
    \item where snails lose infectiousness (assumed to happen sometimes)
    \item with larval population dynamics
    \item single variable models
    \item human immigration and emigration
    \item reservoir hosts
  \end{itemize}
  \vfill
  Really worth a read
\end{frame}


%%%%%%%%%%%%%%%%%%%%
%%%%%%%%%%%%%%%%%%%%
\subsection{A third model of Woolhouse -- Heterogeneous contacts}
\begin{frame}{Heterogeneities in contact rates}
  $I_{i}$ the number of schistosomes in person $i=1,\ldots,H$ and $i_{Sj}$ the proportion of patent infected snails in site $j=1,\ldots,L$ ($L$ sites each supporting $N$ snails)
  \vfill
  \begin{center}
    \def\vertskip{*2}
    \def\horzskip{*1.5}
    \begin{tikzpicture}[scale=1, transform shape]
      \node [circle, fill=gray!10, text=black] at (-1\horzskip,0\vertskip) (site1) {$i_{S1}$};
      \node [circle, fill=gray!10, text=black] at (1\horzskip,0\vertskip) (site2) {$i_{S2}$};
      \node [circle, fill=gray!10, text=black] at (-1\horzskip,1\vertskip) (indiv1) {$I_1$};
      \node [circle, fill=gray!10, text=black] at (0\horzskip,1\vertskip) (indiv2) {$I_2$};
      \node [circle, fill=gray!10, text=black] at (1\horzskip,1\vertskip) (indiv3) {$I_3$};
      %% Flows between
      \path [line, very thick] (indiv1) to node [midway, below] (TextNode) {} (site1);
      \path [line, very thick] (indiv2) to node [near end, above] (TextNode) {} (site1);
      \path [line, very thick] (indiv2) to node [midway, below] (TextNode) {} (site2);
      \path [line, very thick] (indiv3) to node [midway, above] (TextNode) {} (site2);
      % Text
      \node [color=red] at (-2\horzskip,1\vertskip) {Individuals};
      \node [color=red] at (-2\horzskip,0\vertskip) {Sites};
    \end{tikzpicture}    
  \end{center}  
\end{frame}

\begin{frame}
  $I_{i}$ the number of schistosomes in person $i=1,\ldots,H$ and $i_{Sj}$ the proportion of patent infected snails in site $j=1,\ldots,L$ ($L$ sites each supporting $N$ snails)
  \begin{subequations}
    \label{sys:Woolhouse_3}
    \begin{align}
      I_{i}' &= \alpha\left(\sum_j \eta_{ij}Ni_{Sj}\right)-\gamma I_{i} \\
      i_{Sj}' &= \beta\left(\sum_i \eta_{ij}I_i\right)(1-i_{Sj})-\mu_2 i_{Sj}
    \end{align}
  \end{subequations}
  \vfill
  $\eta_{ij}$ rate of water contact by individual $i$ at site $j$
  \vfill  
\end{frame}


%%%%%%%%%%%%%%%%%%%%
%%%%%%%%%%%%%%%%%%%%
\subsection{Spatial aspects -- Cholera in Haiti}
\begin{frame}{Spatial aspects -- Cholera in Haiti}
  Tuite, Tien, Eisenberg, Earn, Ma \& Fisman.
  \href{https://doi.org/10.7326/0003-4819-154-9-201105030-00334}{Cholera Epidemic in Haiti, 2010: Using a Transmission Model to Explain Spatial Spread of Disease and Identify Optimal Control Interventions}. \emph{Annals of Internal Medicine} \textbf{154}(9) (2011)
\end{frame}

\begin{frame}
  \begin{center}
    \def\vertskip{*1}
    \def\horzskip{*4.5}
    \begin{tikzpicture}[scale=0.9, transform shape]
      \node [rectangle, fill=gray!10, text=black] at (0\horzskip,1.5\vertskip) (S_p) {$s_p$};
      \node [rectangle, fill=gray!10, text=black] at (1\horzskip,1.5\vertskip) (I_p) {$i_p$};
      \node [rectangle, fill=gray!10, text=black] at (2\horzskip,1.5\vertskip) (R_p) {$r_p$};
      \node [circle, fill=gray!10, text=black] at (0.5\horzskip,4\vertskip) (W_p) {$W_p$};
      \node [circle, fill=black, color=red, text=white] at (0.5\horzskip,1.5\vertskip) (lambda_p) {$\lambda_ps_p$};
      \node [rectangle, fill=gray!10, text=black] at (0\horzskip,-1.5\vertskip) (S_q) {$s_q$};
      \node [rectangle, fill=gray!10, text=black] at (1\horzskip,-1.5\vertskip) (I_q) {$i_q$};
      \node [rectangle, fill=gray!10, text=black] at (2\horzskip,-1.5\vertskip) (R_q) {$r_q$};
      \node [circle, fill=gray!10, text=black] at (0.5\horzskip,-4\vertskip) (W_q) {$W_q$};
      %% Boxes
      \path [line, rounded corners, thin] (-0.2\horzskip,0.8\vertskip) -- (-0.2\horzskip,4.75\vertskip) -- (2.2\horzskip,4.75\vertskip) -- (2.2\horzskip,0.8\vertskip) -- cycle;
      \path [line, rounded corners, thin] (-0.2\horzskip,-0.8\vertskip) -- (-0.2\horzskip,-4.75\vertskip) -- (2.2\horzskip,-4.75\vertskip) -- (2.2\horzskip,-0.8\vertskip) -- cycle;
      %% Flows in p
      \draw [very thick, color=yellow] (S_p) -- (lambda_p);
      \path [line, very thick] (lambda_p) to node [midway, above] (TextNode) {} (I_p);
      \path [line, very thick] (I_p) to node [midway, above] (TextNode) {$\gamma i_p$} (R_p);
      \path [line, very thick, dashed] (I_p) to node [midway, above] (TextNode) {$\xi i_p$} (W_p);
      \path [line, very thick, dashed] (W_p) to node [midway, left] (TextNode) {$\beta_WW_p$} (lambda_p);
      \path [line, very thick] (W_p) to node [midway, above] (TextNode) {$\xi W_p$} (0.1\horzskip,4\vertskip);
      %% Flows in q
      \path [line, very thick] (S_q) to node [midway, above] (TextNode) {$\lambda_qs_q$} (I_q);
      \path [line, very thick] (I_q) to node [midway, above] (TextNode) {$\gamma i_q$} (R_q);
      \path [line, very thick, dashed] (I_q) to node [midway, above] (TextNode) {$\xi i_q$} (W_q);
      \path [line, very thick] (W_q) to node [midway, above] (TextNode) {$\xi W_q$} (0.1\horzskip,-4\vertskip);
      %% lambda_p
      \path [line, very thick, dashed] (I_q) to node [midway, left] (TextNode) {$\frac{\kappa P_pP_q}{d^n}$} (lambda_p);
      \path [line, very thick, dashed, bend right=50] (I_p) to node [midway, below] (TextNode) {$\beta_Ii_p$} (lambda_p);
      %% Distance between patches
      \draw [line,very thick] (1.5\horzskip,0.8\vertskip) -- (1.5\horzskip,-0.8\vertskip);
      \draw [line,very thick] (1.5\horzskip,-0.8\vertskip) -- (1.5\horzskip,0.8\vertskip);
      \node [color=yellow, text width=3cm] at (1.85\horzskip,0\vertskip) {Distance between regions (d)};
    \end{tikzpicture}    
  \end{center}  
\end{frame}

%\maxFrameImage{../FIGS/Tuite_etal_cholera_model.png}

\begin{frame}
  \textbf{Metapopulation} model with \textbf{implicit} movement
  \begin{subequations}
    \begin{align}
      s_p' &= \mu-\lambda_ps_p-\mu s_p \\
      i_p' &= -\gamma i_p+\lambda_p s_p-\mu i_p \\
      r_p' &= \gamma r_p-\mu r_p \\
      w_p' &= \xi(i_p-w_p)
    \end{align}
    with force of infection
    \begin{equation}
      \lambda_p = \beta_{i_p}i_p+\beta_{W_p}w_p+\sum_{q=1}^{10}\theta_{pq}i_q
    \end{equation}
  \end{subequations}
  \vfill
  Influence of infection prevalence in $q$ on incidence in $p$ is gravity-type
  \[
    \theta_{pq}=\kappa\frac{P_pP_q}{d^n}
  \]
\end{frame}

\maxFrameImage{../FIGS/334ff2.jpeg}

% \begin{frame}{A few other interesting papers}
%   \begin{itemize}
%     \item Joel E. Cohen. \href{https://www.jstor.org/stable/2096727}{Mathematical Models of Schistosomiasis}. \emph{Annual Review of Ecology and Systematics} \textbf{8}:209-233 (1977) 
%   \end{itemize}
% \end{frame}


\end{document}