\documentclass[aspectratio=169]{beamer}

%\documentclass[handout]{beamer}
%% To make 4 per page
%\usepackage{pgfpages}
%\mode<handout>{\setbeamercolor{background canvas}{bg=white}}
%\pgfpagesuselayout{4 on 1}[letterpaper,landscape]%,border shrink=5mm]

\usetheme{default}
\usepackage{bm}
\usepackage{colortbl}

\usepackage{amsmath,amssymb,amsthm}
%\usepackage{subfigure}


% Fields and the like
\def\IC{\mathbb{C}}
\def\IF{\mathbb{F}}
\def\II{\mathbb{I}}
\def\IM{\mathbb{M}}
\def\IN{\mathbb{N}}
\def\IP{\mathbb{P}}
\def\IR{\mathbb{R}}
\def\IZ{\mathbb{Z}}

% Bold lowercase
\def\ba{\mathbf{a}}
\def\bb{\mathbf{b}}
\def\bc{\mathbf{c}}
\def\bd{\mathbf{d}}
\def\be{\mathbf{e}}
\def\bf{\mathbf{f}}
\def\bh{\mathbf{h}}
\def\bi{\mathbf{i}}
\def\bj{\mathbf{j}}
\def\bk{\mathbf{k}}
\def\bn{\mathbf{n}}
\def\bp{\mathbf{p}}
\def\br{\mathbf{r}}
\def\bs{\mathbf{s}}
\def\bu{\mathbf{u}}
\def\bv{\mathbf{v}}
\def\bw{\mathbf{w}}
\def\bx{\mathbf{x}}
\def\by{\mathbf{y}}
\def\bz{\mathbf{z}}

% Bold capitals
\def\bB{\mathbf{B}}
\def\bD{\mathbf{D}}
\def\bF{\mathbf{F}}
\def\bG{\mathbf{G}}
\def\bI{\mathbf{I}}
\def\bL{\mathbf{L}}
\def\bN{\mathbf{N}}
\def\bR{\mathbf{R}}
\def\bS{\mathbf{S}}
\def\bT{\mathbf{T}}
\def\bX{\mathbf{X}}

% Bold numbers
\def\b0{\mathbf{0}}

% Bold greek
\bmdefine{\bmu}{\bm{\mu}}
\def\bphi{\bm{\phi}}
\def\bvarphi{\bm{\varphi}}

% Bold red sentence
\def\boldred#1{{\color{red}\textbf{#1}}}
\def\defword#1{{\color{orange}\textbf{#1}}}

% Caligraphic letters
\def\A{\mathcal{A}}
\def\B{\mathcal{B}}
\def\C{\mathcal{C}}
\def\D{\mathcal{D}}
\def\E{\mathcal{E}}
\def\F{\mathcal{F}}
\def\G{\mathcal{G}}
\def\I{\mathcal{I}}
\def\L{\mathcal{L}}
\def\M{\mathcal{M}}
\def\P{\mathcal{P}}
\def\R{\mathcal{R}}
\def\S{\mathcal{S}}
\def\T{\mathcal{T}}
\def\U{\mathcal{U}}
\def\V{\mathcal{V}}

% tt font for code
\def\code#1{{\tt #1}}

% Operators and special symbols
\def\nbOne{{\mathchoice {\rm 1\mskip-4mu l} {\rm 1\mskip-4mu l}
{\rm 1\mskip-4.5mu l} {\rm 1\mskip-5mu l}}}
\def\cov{\ensuremath{\mathsf{cov}}}
\def\Var{\ensuremath{\mathsf{Var}\ }}
\def\Im{\textrm{Im}\;}
\def\Re{\textrm{Re}\;}
\def\det{\ensuremath{\mathsf{det}}}
\def\diag{\ensuremath{\mathsf{diag}}}
\def\nullspace{\ensuremath{\mathsf{null}}}
\def\nullity{\ensuremath{\mathsf{nullity}}}
\def\rank{\ensuremath{\mathsf{rank}}}
\def\range{\ensuremath{\mathsf{range}}}
\def\sgn{\ensuremath{\mathsf{sgn}}}
\def\Span{\ensuremath{\mathsf{span}}}
\def\tr{\ensuremath{\mathsf{tr}}}
\def\imply{$\Rightarrow$}
\def\restrictTo#1#2{\left.#1\right|_{#2}}
\newcommand{\parallelsum}{\mathbin{\!/\mkern-5mu/\!}}

% The beamer bullet (in base colour)
\def\bbullet{\leavevmode\usebeamertemplate{itemize item}\ }

% Theorems and the like
\newtheorem{proposition}[theorem]{Proposition}
\newtheorem{property}[theorem]{Property}
\newtheorem{importantproperty}[theorem]{Property}
\newtheorem{importanttheorem}[theorem]{Theorem}
%\newtheorem{lemma}[theorem]{Lemma}
%
%\usecolortheme{orchid}
\setbeamertemplate{theorems}[numbered]
%\usecolortheme{orchid}
%\setbeamertemplate{theorems}[ams style]
%\setbeamertemplate{theorems}[numbered]

%% Listings
\usepackage{listings}
\definecolor{mygreen}{rgb}{0,0.6,0}
\definecolor{mygray}{rgb}{0.5,0.5,0.5}
\definecolor{mymauve}{rgb}{0.58,0,0.82}
\definecolor{mygold}{rgb}{1,0.843,0}
\definecolor{myblue}{rgb}{0.537,0.812,0.941}

\definecolor{lgreen}{rgb}{0.6,0.9,.6}
\definecolor{lred}{rgb}{1,0.5,.5}

\lstloadlanguages{R}
\lstset{ %
  language=R,
  backgroundcolor=\color{black!95},   % choose the background color
  basicstyle=\footnotesize\ttfamily,        % size of fonts used for the code
  breaklines=true,                 % automatic line breaking only at whitespace
  captionpos=b,                    % sets the caption-position to bottom
  commentstyle=\color{mygreen},    % comment style
  escapeinside={\%*}{*)},          % if you want to add LaTeX within your code
  keywordstyle=\color{myblue},       % keyword style
  stringstyle=\color{mygold},     % string literal style
  keepspaces=true,
  columns=fullflexible,
  tabsize=4,
}
% Could also do (in lstset)
% basicstyle==\fontfamily{pcr}\footnotesize


% Get rid of navigation stuff
\setbeamertemplate{navigation symbols}{}

% Set footline/header line
\setbeamertemplate{footline}
{%
\quad p. \insertpagenumber \quad--\quad \insertsection\vskip2pt
}
% \setbeamertemplate{headline}
% {%
% \quad\insertsection\hfill p. \insertpagenumber\quad\mbox{}\vskip2pt
% }


\makeatletter
\newlength\beamerleftmargin
\setlength\beamerleftmargin{\Gm@lmargin}
\makeatother

%%%%%%% 
%% Definitions in yellow boxes
\usepackage{etoolbox}
\setbeamercolor{block title}{use=structure,fg=structure.fg,bg=structure.fg!05!bg}
\setbeamercolor{block body}{parent=normal text,use=block title,bg=block title.bg!20!bg}

\BeforeBeginEnvironment{definition}{%
	\setbeamercolor{block title}{fg=black,bg=yellow!20!white}
	\setbeamercolor{block body}{fg=black, bg=yellow!05!white}
}
\AfterEndEnvironment{definition}{
	\setbeamercolor{block title}{use=structure,fg=structure.fg,bg=structure.fg!20!bg}
	\setbeamercolor{block body}{parent=normal text,use=block title,bg=block title.bg!50!bg, fg=black}
}
\BeforeBeginEnvironment{importanttheorem}{%
	\setbeamercolor{block title}{fg=black,bg=red!20!white}
	\setbeamercolor{block body}{fg=black, bg=red!05!white}
}
\AfterEndEnvironment{importanttheorem}{
	\setbeamercolor{block title}{use=structure,fg=structure.fg,bg=structure.fg!20!bg}
	\setbeamercolor{block body}{parent=normal text,use=block title,bg=block title.bg!50!bg, fg=black}
}
\BeforeBeginEnvironment{theorem}{%
	\setbeamercolor{block title}{fg=white,bg=red!30!black}
	\setbeamercolor{block body}{fg=white, bg=red!10!black}
}
\AfterEndEnvironment{theorem}{
	\setbeamercolor{block title}{use=structure,fg=structure.fg,bg=structure.fg!20!bg}
	\setbeamercolor{block body}{parent=normal text,use=block title,bg=block title.bg!50!bg, fg=black}
}
\BeforeBeginEnvironment{importantproperty}{%
	\setbeamercolor{block title}{fg=black,bg=red!50!white}
	\setbeamercolor{block body}{fg=black, bg=red!30!white}
}
\AfterEndEnvironment{importantproperty}{
	\setbeamercolor{block title}{use=structure,fg=structure.fg,bg=structure.fg!20!bg}
	\setbeamercolor{block body}{parent=normal text,use=block title,bg=block title.bg!50!bg, fg=black}
}



%%%%%%%%%%%%%%%%%
\usepackage{tikz}
\usetikzlibrary{shapes,arrows}
\usetikzlibrary{positioning}
\usetikzlibrary{shapes.symbols,shapes.callouts,patterns}
\usetikzlibrary{calc,fit}
\usetikzlibrary{backgrounds}
\usetikzlibrary{decorations.pathmorphing,fit,petri}
\usetikzlibrary{automata}
\usetikzlibrary{fadings}
\usetikzlibrary{patterns,hobby}

\usepackage{pgfplots}
\pgfplotsset{compat=1.6}
\pgfplotsset{ticks=none}

\usetikzlibrary{decorations.markings}
\usetikzlibrary{arrows.meta}
\tikzset{>=stealth}

\tikzstyle{cloud} = [draw, 
ellipse,
fill=red!20, 
node distance=0.87cm,
minimum height=2em]
\tikzstyle{line} = [draw, 
-latex', 
color=yellow]


% Beginning of a section
% \AtBeginSection[]{
% 	{
% 		\setbeamercolor{background canvas}{bg=orange!10}
% 		\begin{frame}[noframenumbering,plain]
% 			\framesubtitle{\nameofthepart Chapter \insertromanpartnumber \ -- \iteminsert{\insertpart}}
% 			\tableofcontents[currentsection,currentsubsection]
% 		\end{frame}
% 	\addtocounter{page}{-1}
% 	%\addtocounter{framenumber}{-1} 
% 	}
% }


%%% SLIDES COLOURING

%\usecolortheme{owl}

\setbeamerfont{frametitle}{series=\bfseries}
\setbeamercolor{frametitle}{fg=black!05,bg=black}

\setbeamerfont{framesubtitle}{size=\normalfont\tiny}
\setbeamercolor{framesubtitle}{fg=black!05}

\setbeamercolor{background canvas}{bg=black}
\setbeamercolor{normal text}{fg=black!10}



\definecolor{bottomcolour}{rgb}{0.32,0.3,0.38}
\definecolor{middlecolour}{rgb}{0.08,0.08,0.16}
\definecolor{mycolor}{rgb}{0.4,0.4, 0.4}
% Beginning of a section
\AtBeginSection[]{
	{
		%\setbeamercolor{background canvas}[vertical shading][top=bottomcolour, middle=middlecolour, bottom=black]
		\setbeamertemplate{background canvas}[vertical shading][bottom=bottomcolour,top=black!20]
    % \setbeamertemplate{background canvas}{
    %   \begin{tikzpicture}%[remember picture,overlay]
    %     \shade[top color=yellow!75!green!33,
    %     bottom color=blue!66!green!33,
    %     middle color=blue!6!green!33]
    %   \end{tikzpicture}
    % }
    \begin{frame}[noframenumbering,plain]
			\framesubtitle{\nameofthepart Chapter \insertromanpartnumber \ -- \iteminsert{\insertpart}}
			\tableofcontents[currentsection,currentsubsection]
		\end{frame}
	\addtocounter{page}{-1}
	}
}
% Beginning of a section
\AtBeginSubsection[]{
	{
		%\setbeamercolor{background canvas}[vertical shading][top=bottomcolour, middle=middlecolour, bottom=black]
		%\setbeamertemplate{background canvas}[vertical shading][bottom=bottomcolour,top=black!20]
    \setbeamertemplate{background canvas}{
      \begin{tikzpicture}%[remember picture,overlay]
        \shade[top color=yellow!75!green!33,
        bottom color=blue!66!green!33,
        middle color=blue!6!green!33]
      \end{tikzpicture}
    }
    \begin{frame}[noframenumbering,plain]
			\framesubtitle{\nameofthepart Chapter \insertromanpartnumber \ -- \iteminsert{\insertpart}}
			\tableofcontents[currentsection,currentsubsection]
		\end{frame}
	\addtocounter{page}{-1}
	}
}

% Colours for special pages
\def\extraContent{yellow!20}

%% Allow to change slide colour
%% From: https://tex.stackexchange.com/questions/8043/change-the-background-color-of-a-frame-in-beamer
\defbeamertemplate*{background canvas}{mydefault}{%
  \ifbeamercolorempty[bg]{background canvas}{}{\color{bg}\vrule width\paperwidth height\paperheight}% copied beamer default here
}
\defbeamertemplate*{background canvas}{bg}{%
  \color{lightgray!20}\vrule width\paperwidth height\paperheight% added bg color
}
\BeforeBeginEnvironment{frame}{%
  \setbeamertemplate{background canvas}[mydefault]%
}
\makeatletter
\define@key{beamerframe}{bg}[true]{%
  \setbeamertemplate{background canvas}[bg]%
}
\makeatother
% Use with
%\begin{frame}
% \frametitle{Normal}
%\end{frame} 
%\begin{frame}[bg]
% \frametitle{With bg}
%\end{frame}


%% Vertical alignment on pages
%% From: https://tex.stackexchange.com/questions/148365/how-do-i-ask-beamer-to-exactly-fill-up-a-slide
%% Turn on with
%% \stretchon
%% (outside slide), and off with
%% \stretchoff
% \def\itemsymbol{$\blacktriangleright$}
% \let\svpar\par
% \let\svitemize\itemize
% \let\svenditemize\enditemize
% \let\svitem\item
% \let\svcenter\center
% \let\svendcenter\endcenter
% \let\svcolumn\column
% \let\svendcolumn\endcolumn
% \def\newitem{\renewcommand\item[1][\itemsymbol]{\vfill\svitem[##1]}}%
% \def\newpar{\def\par{\svpar\vfill}}%
% \newcommand\stretchon{%
%   \newpar%
%   \renewcommand\item[1][\itemsymbol]{\svitem[##1]\newitem}%
%   \renewenvironment{itemize}%
%     {\svitemize}{\svenditemize\newpar\par}%
%   \renewenvironment{center}%
%     {\svcenter\newpar}{\svendcenter\newpar}%
%   \renewenvironment{column}[2]%
%     {\svcolumn{##1}\setlength{\parskip}{\columnskip}##2}%
%     {\svendcolumn\vspace{\columnskip}}%
% }
% \newcommand\stretchoff{%
%   \let\par\svpar%
%   \let\item\svitem%
%   \let\itemize\svitemize%
%   \let\enditemize\svenditemize%
%   \let\center\svcenter%
%   \let\endcenter\svendcenter%
%   \let\column\svcolumn%
%   \let\endcolumn\svendcolumn%
% }
% \newlength\columnskip
% \columnskip 0pt


\newenvironment{changemargin}[2]{%
\begin{list}{}{%
\setlength{\topsep}{0pt}%
\setlength{\leftmargin}{#1}%
\setlength{\rightmargin}{#2}%
\setlength{\listparindent}{\parindent}%
\setlength{\itemindent}{\parindent}%
\setlength{\parsep}{\parskip}%
}%
\item[]}{\end{list}}


% Make one image take up the entire slide content area in beamer,.:
% centered/centred full-screen image, with title:
% This uses the whole screen except for the 1cm border around it
% all. 128x96mm
\newcommand{\titledFrameImage}[2]{
\begin{frame}{#1}
%\begin{changemargin}{-1cm}{-1cm}
\begin{center}
\includegraphics[width=108mm,height=\textheight,keepaspectratio]{#2}
\end{center}
%\end{changemargin}
\end{frame}
}

% Make one image take up the entire slide content area in beamer.:
% centered/centred full-screen image, no title:
% This uses the whole screen except for the 1cm border around it
% all. 128x96mm
\newcommand{\plainFrameImage}[1]{
\begin{frame}[plain]
%\begin{changemargin}{-1cm}{-1cm}
\begin{center}
\includegraphics[width=108mm,height=76mm,keepaspectratio]{#1}
\end{center}
%\end{changemargin}
\end{frame}
}

% Make one image take up the entire slide area, including borders, in beamer.:
% centered/centred full-screen image, no title:
% This uses the entire whole screen
\newcommand{\maxFrameImage}[1]{
\begin{frame}[plain]
\begin{changemargin}{-1cm}{-1cm}
\begin{center}
\includegraphics[width=\paperwidth,height=\paperheight,keepaspectratio]
{#1}
\end{center}
\end{changemargin}
\end{frame}
}

% Make one image take up the entire slide area, including borders, in beamer.:
% centered/centred full-screen image, no title:
% This uses the entire whole screen
\newcommand{\maxFrameImageColor}[2]{
\begin{frame}[plain]
\setbeamercolor{normal text}{bg=#2!20}
\begin{changemargin}{-1cm}{-1cm}
\begin{center}
\includegraphics[width=\paperwidth,height=\paperheight,keepaspectratio]
{#1}
\end{center}
\end{changemargin}
\end{frame}
}



\title{Environmentally Transmitted Pathogens}
\subtitle{Models}
\author{Julien Arino}
\date{January 2023}


\begin{document}
%\stretchon

% The title page
\begin{frame}[noframenumbering,plain]
  \titlepage
\end{frame}
\addtocounter{page}{-1}



%%%%%%%%%%%%%%%%%%%%%%%%%%
%%%%%%%%%%%%%%%%%%%%%%%%%%
%%%%%%%%%%%%%%%%%%%%%%%%%%
%%%%%%%%%%%%%%%%%%%%%%%%%%
\section{Fitting}

%%%%%%%%%%%%%%%%%%%%%%%%%%
%%%%%%%%%%%%%%%%%%%%%%%%%%
\subsection{General principles}
\begin{frame}{Fitting a model to parameters}
Very simplified version of what is done in practice

There are way more elaborate methods. See, e.g., 
\begin{itemize}
    \item Roda. \href{https://doi.org/10.1016/j.idm.2019.12.007}{Bayesian inference for dynamical systems}
    \item Portet. \href{https://doi.org/10.1016/j.idm.2019.12.010}{A primer on model selection using the Akaike Information Criterion}
\end{itemize}
\end{frame}

\begin{frame}{Parameter fitting problem}
Assume given data points $(t_i,y_i)$, $i=1,\ldots,N$, with $y_i\in\IR^n$ and $t_i\in\I\subset\IR$, where $\I$ is some interval of time
\vfill
Assume model parameters are  in a set $\P$
\vfill
Solution to the ODE is $x(t,p)$ for $t\in\I$ and a given $p\in\P$ (we emphasise the dependent of the solution on the chosen point in parameter space)
\vfill
{
    \setbeamercolor{block title}{bg=red!60, fg=black}
    \setbeamercolor{block body}{bg=red!10, fg=black}
    \begin{block}{Parameter fitting problem}
        Find $p\in\P$ such that the solution $x(t,p)$ ``most closely matches'' the data points
    \end{block}    
}
\end{frame}

\begin{frame}
    Could also be another type of system (discrete-time, continuous-time Markov chain -- in which case we would likely use mean solution over a number of realisations, etc.), but in any event, a time series somewhat comparable with the data 
\end{frame}

\begin{frame}{Mathematical formulation}
We want to minimise the error function
\begin{equation}\label{eq:general_error}
    E(p) = \sum_{i=1}^N \|h(x(t_i))-y_i\|
\end{equation}
\begin{itemize}
    \item $h:\IR^n\to\IR^n$ is an \textbf{observation function}, which selects the relevant part of the solution to match to the data
    \item The norm is usually the Euclidean norm, but could be different depending on the problem at hand
    \item Given a parametre $p$ in an (admissible part) of parameter space, compute the solution to the ODE, then deduce $E(p)$
    \item Use a minimisation algorithm to find a minimum of $E(p)$ while varying $p$  
\end{itemize}
\end{frame}



\begin{frame}{What are $y_i$ and $h(x(t_i))$ here?}
    Infectious disease data is ofen in terms of \emph{incidence} (number of new cases per unit time) rather than \emph{prevalence} (number of cases currently present in the population)
    \vfill
    In, say, an SIR model with mass action incidence, incidence is $\beta SI$, so, using the Euclidean norm
    \begin{equation}
        E(p)=\sum_{i=1}^N(\beta S(t_i)I(t_i)-y_i)^2
    \end{equation}
    where $S(t_i)$ and $I(t_i)$ are the values of the numerical solution to the ODE at the times $t_i$ in the data 
\end{frame}


%%%%%%%%%%%%%%%%%%%%%%%%%%
%%%%%%%%%%%%%%%%%%%%%%%%%%
\subsection{A simple example -- The logistic function}

\begin{frame}{}
    To illustrate the method, let us use  a simple case and fit the logistic equation $N'=rN(1-N/K)$ to some population data
    \vfill 
    We use El Salvador, whose population seems to be experiencing growth slowdown. We get population data from the World Bank
    \vfill
    We seek values of $r$ and $K$ that minimise $E(p)$ given by \eqref{eq:general_error} with norm the Euclidean norm
\end{frame}

\begin{frame}[fragile]{The right hand side}
This one is quite simple, of course\ldots
\vfill
\begin{lstlisting}
RHS_logistic <- function(t, x, p) {
    with(as.list(c(x,p)), {
    dN <- r*N*(1-N/K)
    return(list(dN))
    })
}    
\end{lstlisting}
\end{frame}

\begin{frame}[fragile]{Getting the population data}
\begin{lstlisting}
library(wbstats)
get_pop_data <- function(CTRY) {
    pop = wb_data("SP.POP.TOTL", country = CTRY,
                mrv = 100, return_wide = FALSE)
    pop = pop[,c("date", "value")]
    pop = pop[order(pop$date),]
    pop$date = as.numeric(pop$date)
    pop$value = as.numeric(pop$value)
    return(pop)
}    
\end{lstlisting}
which we call later using
\begin{lstlisting}
pop = get_pop_data("El Salvador")
\end{lstlisting}
\vfill
Note that if you do not set \code{return\_wide=TRUE}, the column with the return value is named like the indicator (\code{SP.POP.TOTL} here), otherwise it is \code{value}
\end{frame}


\begin{frame}{The error function}
    This is the function that does most of the work
    \vfill
    \begin{itemize}
        \item Takes as input the parameters that vary and any other required parameters
        \item Computes the solution of the ODE
        \begin{itemize}
            \item Capitalise on the fact that \code{ode} returns the solution at \code{times} that you can specify!
            \item If your data is at times, say, $t=1,2,5,10$, then you can pass \code{times=c(1,2,5,10)} and get the solution at these times, making the next step easy
        \end{itemize}
        \item Compute the error
    \end{itemize}
    \vfill
    Actually, this does not do most of the work but this is definitely where \emph{you} need to do most of the work
\end{frame}

\begin{frame}[fragile]
\begin{lstlisting}
error_fit <- function(p_vary, 
                      params, 
                      data,
                      method = "rk4") {
    # Anything that changes during optimisation is set here
    params$r = as.numeric(p_vary["r"])
    params$K = as.numeric(p_vary["K"])
    # Set the initial condition
    N0 = data$value[1]
    IC = c(N = N0)
    # Compute the solution
    sol = ode(y = IC, times = data$date, func = RHS_logistic, 
              parms = params, method = method)
\end{lstlisting}
\end{frame}

\begin{frame}[fragile]{One little trick}
    If a parameter value or a solution is ``not acceptable'', one easy way to deal with this is to return an error value of \code{Inf} (i.e., $\infty$) and immediately exit the error function
    \vfill
\begin{lstlisting}
    if (sol[dim(sol)[1],"time"] < data$date[length(data$date)]) {
        return(Inf)
    }    
\end{lstlisting}
would be triggered if, for instance, the numerical integration finished early (because solutions explode, e.g.)
\vfill
Useful also, e.g., if you want to exclude regions in parameter space. Say you want to find parameters s.t. $\R_0\leq 10$, then you could return an \code{Inf} error whenever parameters are s.t. $\R_0>10$
\end{frame}


\begin{frame}[fragile]{Computing the actual error}
\begin{lstlisting}[numbers=left]
    diff_values = data$value - sol[,"N"]
    diff_values_squared = diff_values^2
    error = sum(diff_values_squared)
    return(error)
}  # END error_incidence
\end{lstlisting}
\vfill
Line 1: here, $h(x)=x$, we can simply use $N(t)$ as the observed variable. So we compute the $y_i-N(t_i)$, since we have set \code{times=data\$date} and thus have matching time points
\vfill
Line 2: square the values (recall that by default, \code{R} does Hadamard-type operations, so here, squares each entry in the vector), giving a vector with entries $(y_i-N(t_i))^2$
\vfill
Line 3: sum the elements of the vector, i.e., obtain $\sum_{i}(y_i-N(t_i))^2$
\end{frame}

\begin{frame}[fragile]{Set up a last few things}
Back in the main code, set values for the parameters, although they will be changed by the optimiser
\begin{lstlisting}
params = list()
params$r = 1
params$K = max(pop$value)
\end{lstlisting}
\end{frame}

\begin{frame}[fragile]{Now let an optimiser do the actual work}
Here, let us use a genetic algorithm
\begin{lstlisting}
library(GA)
GA = ga(
  type = "real-valued",
  fitness = function(x) 
    -error_fit(p_vary = c(r = x[1], K = x[2]),
               params = params,
               data = pop,
               method = "rk4"),
  parallel = TRUE,
  lower = c(0.1, 1000000),
  upper = c(10, 10000000),
  optim = TRUE,
  optimArgs = list(method = "CG"),
  suggestions = c(1, params$K),
  popSize = 500,
  maxiter = 200
)
\end{lstlisting}
\end{frame}

\begin{frame}{Explaining the algorithm and the call to \code{GA}}
    A \textbf{genetic algorithm} (GA) mimics evolution selecting for increased fitness.
    \vfill
    \begin{itemize}
        \item A \textbf{gene} is a point $p^\star\in\P$ in parameter space, so, here, a given value $(r^\star,K^\star)$ of $(r,K)$
        \item The \textbf{fitness} of the gene is the value of the function to optimise when evaluated at $p^\star$, i.e., here, $E(p^\star)$ (i.e., \code{error\_fit})
        \item A GA ``wants to'' maximise fitness, so we actually use $-E(p)$
    \end{itemize}
\end{frame}

\begin{frame}{Setting up the gene pool}
    \begin{itemize}
        \item Start with a randomly selected population of \code{popSize} genes (500 here)
        \item Within these \code{popSize} genes, one (could be more) is a \code{suggestion}. Here, I have taken \code{r=1} and \code{K=params\$K=max(pop\$value)}
        \item Genes other than the suggested ones are selected at random (potentially following some distribution, but by default and here, uniformly) between \code{lower} and \code{upper}
        \item  (The order of elements of the gene is important in some places, including \code{suggestions}, \code{upper} and \code{lower})
    \end{itemize}
\end{frame}

\begin{frame}
At each iteration, up to a maximum \code{maxiter} (200 here)
\begin{itemize}
    \item Compute the fitness of all \code{popSize} genes
    \item Retain the genes with highest fitness
    \item Throw in new genes, some at random like before but others using ``constrained randomness'', i.e., using analogues of genetic operations such as mutations, crossovers, etc.
    \item Once the next generation is ready, i.e., there are \code{popSize} genes, restart
\end{itemize}
\vfill
It is possible to specify how much of the existing gene pool to keep, etc.
\end{frame}

\begin{frame}{Two interesting options}
\begin{enumerate}
    \item \code{parallel=TRUE} (needs libraries \code{parallel} and \code{doParallel}) parallelises the code. Each function (fitness) evaluation is completely independent from others, so this algorithm parallelises very nicely, leading to potentially consequential speedups
    \item \code{optim=TRUE} interrupts the GA execution (including parallel component if used) to perform a step of deterministic gradient descent search close to the best value found so far, in case there is no change in best value for a few generation. This allows to potentially fine tune a stochastically found optimum
\end{enumerate}
\vfill
\textbf{Note --} If you want to limit the number of threads used (to avoid completely bogging down your computer), you can specify a number of threads to use, instead, e.g., \code{parallel=10}. Currently, you also \emph{must} specify \code{parallel=124} if you have a CPU with more than 124 threads
\end{frame}

\maxFrameImage{../FIGS/fit_logistic_1}


\begin{frame}[fragile]{Varying also $N_0$}
    Previous graph fits precisely the first data point. We can want to also fit $N_0$
    \vfill
    In the function \code{error\_incidence}, add
\begin{lstlisting}
params$N0 = as.numeric(p_vary["N0"])
\end{lstlisting}
then in the call to \code{error\_incidence} in \code{ga},
\begin{lstlisting}
fitness = function(x)
    -error_incidence(p_vary = c(r = x[1], K = x[2], N0 = x[3]),
                     params = params,
                     data = pop,
                     method = "rk4"),
\end{lstlisting}
We get a smaller error and something like on the next page
\end{frame}

\maxFrameImage{../FIGS/fit_logistic_2}

%%%%%%%%%%%%%%%%%%%%%%%%%%
%%%%%%%%%%%%%%%%%%%%%%%%%%
\subsection{The issue of parameter identifiability}

\begin{frame}{}
    See, e.g., Roda \emph{et al}: \href{https://doi.org/10.1016/j.idm.2020.03.001}{Why is it difficult to accurately predict the COVID-19 epidemic?}
    \vfill
    We seek to minimise the (error) function
    \begin{equation}\tag{\ref{eq:general_error}}
        E(p) = \sum_{i=1}^N \|h(x(t_i))-y_i\|
    \end{equation}
    \vfill
    \begin{itemize}
        \item It is possible (extremely likely with complex models) that several values of $p$ minimise $E(p)$
        \item It is also possible that the value(s) found for $p$ are only \emph{local} minima 
        \item These problems are linked to the so-called \textbf{identifiability} problem
    \end{itemize}
\end{frame}


\end{document}