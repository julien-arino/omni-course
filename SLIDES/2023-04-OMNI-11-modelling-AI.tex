\documentclass[aspectratio=43]{beamer}

%\documentclass[handout]{beamer}
%% To make 4 per page
%\usepackage{pgfpages}
%\mode<handout>{\setbeamercolor{background canvas}{bg=white}}
%\pgfpagesuselayout{4 on 1}[letterpaper,landscape]%,border shrink=5mm]

\usetheme{default}
\usepackage{bm}
\usepackage{colortbl}

\usepackage{amsmath,amssymb,amsthm}
%\usepackage{subfigure}


% Fields and the like
\def\IC{\mathbb{C}}
\def\IF{\mathbb{F}}
\def\II{\mathbb{I}}
\def\IM{\mathbb{M}}
\def\IN{\mathbb{N}}
\def\IP{\mathbb{P}}
\def\IR{\mathbb{R}}
\def\IZ{\mathbb{Z}}

% Bold lowercase
\def\ba{\mathbf{a}}
\def\bb{\mathbf{b}}
\def\bc{\mathbf{c}}
\def\bd{\mathbf{d}}
\def\be{\mathbf{e}}
\def\bf{\mathbf{f}}
\def\bh{\mathbf{h}}
\def\bi{\mathbf{i}}
\def\bj{\mathbf{j}}
\def\bk{\mathbf{k}}
\def\bn{\mathbf{n}}
\def\bp{\mathbf{p}}
\def\br{\mathbf{r}}
\def\bs{\mathbf{s}}
\def\bu{\mathbf{u}}
\def\bv{\mathbf{v}}
\def\bw{\mathbf{w}}
\def\bx{\mathbf{x}}
\def\by{\mathbf{y}}
\def\bz{\mathbf{z}}

% Bold capitals
\def\bB{\mathbf{B}}
\def\bD{\mathbf{D}}
\def\bF{\mathbf{F}}
\def\bG{\mathbf{G}}
\def\bI{\mathbf{I}}
\def\bL{\mathbf{L}}
\def\bN{\mathbf{N}}
\def\bR{\mathbf{R}}
\def\bS{\mathbf{S}}
\def\bT{\mathbf{T}}
\def\bX{\mathbf{X}}

% Bold numbers
\def\b0{\mathbf{0}}

% Bold greek
\bmdefine{\bmu}{\bm{\mu}}
\def\bphi{\bm{\phi}}
\def\bvarphi{\bm{\varphi}}

% Bold red sentence
\def\boldred#1{{\color{red}\textbf{#1}}}
\def\defword#1{{\color{orange}\textbf{#1}}}

% Caligraphic letters
\def\A{\mathcal{A}}
\def\B{\mathcal{B}}
\def\C{\mathcal{C}}
\def\D{\mathcal{D}}
\def\E{\mathcal{E}}
\def\F{\mathcal{F}}
\def\G{\mathcal{G}}
\def\I{\mathcal{I}}
\def\L{\mathcal{L}}
\def\M{\mathcal{M}}
\def\P{\mathcal{P}}
\def\R{\mathcal{R}}
\def\S{\mathcal{S}}
\def\T{\mathcal{T}}
\def\U{\mathcal{U}}
\def\V{\mathcal{V}}

% tt font for code
\def\code#1{{\tt #1}}

% Operators and special symbols
\def\nbOne{{\mathchoice {\rm 1\mskip-4mu l} {\rm 1\mskip-4mu l}
{\rm 1\mskip-4.5mu l} {\rm 1\mskip-5mu l}}}
\def\cov{\ensuremath{\mathsf{cov}}}
\def\Var{\ensuremath{\mathsf{Var}\ }}
\def\Im{\textrm{Im}\;}
\def\Re{\textrm{Re}\;}
\def\det{\ensuremath{\mathsf{det}}}
\def\diag{\ensuremath{\mathsf{diag}}}
\def\nullspace{\ensuremath{\mathsf{null}}}
\def\nullity{\ensuremath{\mathsf{nullity}}}
\def\rank{\ensuremath{\mathsf{rank}}}
\def\range{\ensuremath{\mathsf{range}}}
\def\sgn{\ensuremath{\mathsf{sgn}}}
\def\Span{\ensuremath{\mathsf{span}}}
\def\tr{\ensuremath{\mathsf{tr}}}
\def\imply{$\Rightarrow$}
\def\restrictTo#1#2{\left.#1\right|_{#2}}
\newcommand{\parallelsum}{\mathbin{\!/\mkern-5mu/\!}}

% The beamer bullet (in base colour)
\def\bbullet{\leavevmode\usebeamertemplate{itemize item}\ }

% Theorems and the like
\newtheorem{proposition}[theorem]{Proposition}
\newtheorem{property}[theorem]{Property}
\newtheorem{importantproperty}[theorem]{Property}
\newtheorem{importanttheorem}[theorem]{Theorem}
%\newtheorem{lemma}[theorem]{Lemma}
%
%\usecolortheme{orchid}
\setbeamertemplate{theorems}[numbered]
%\usecolortheme{orchid}
%\setbeamertemplate{theorems}[ams style]
%\setbeamertemplate{theorems}[numbered]

%% Listings
\usepackage{listings}
\definecolor{mygreen}{rgb}{0,0.6,0}
\definecolor{mygray}{rgb}{0.5,0.5,0.5}
\definecolor{mymauve}{rgb}{0.58,0,0.82}
\definecolor{mygold}{rgb}{1,0.843,0}
\definecolor{myblue}{rgb}{0.537,0.812,0.941}

\definecolor{lgreen}{rgb}{0.6,0.9,.6}
\definecolor{lred}{rgb}{1,0.5,.5}

\lstloadlanguages{R}
\lstset{ %
  language=R,
  backgroundcolor=\color{black!95},   % choose the background color
  basicstyle=\footnotesize\ttfamily,        % size of fonts used for the code
  breaklines=true,                 % automatic line breaking only at whitespace
  captionpos=b,                    % sets the caption-position to bottom
  commentstyle=\color{mygreen},    % comment style
  escapeinside={\%*}{*)},          % if you want to add LaTeX within your code
  keywordstyle=\color{myblue},       % keyword style
  stringstyle=\color{mygold},     % string literal style
  keepspaces=true,
  columns=fullflexible,
  tabsize=4,
}
% Could also do (in lstset)
% basicstyle==\fontfamily{pcr}\footnotesize


% Get rid of navigation stuff
\setbeamertemplate{navigation symbols}{}

% Set footline/header line
\setbeamertemplate{footline}
{%
\quad p. \insertpagenumber \quad--\quad \insertsection\vskip2pt
}
% \setbeamertemplate{headline}
% {%
% \quad\insertsection\hfill p. \insertpagenumber\quad\mbox{}\vskip2pt
% }


\makeatletter
\newlength\beamerleftmargin
\setlength\beamerleftmargin{\Gm@lmargin}
\makeatother

%%%%%%% 
%% Definitions in yellow boxes
\usepackage{etoolbox}
\setbeamercolor{block title}{use=structure,fg=structure.fg,bg=structure.fg!05!bg}
\setbeamercolor{block body}{parent=normal text,use=block title,bg=block title.bg!20!bg}

\BeforeBeginEnvironment{definition}{%
	\setbeamercolor{block title}{fg=black,bg=yellow!20!white}
	\setbeamercolor{block body}{fg=black, bg=yellow!05!white}
}
\AfterEndEnvironment{definition}{
	\setbeamercolor{block title}{use=structure,fg=structure.fg,bg=structure.fg!20!bg}
	\setbeamercolor{block body}{parent=normal text,use=block title,bg=block title.bg!50!bg, fg=black}
}
\BeforeBeginEnvironment{importanttheorem}{%
	\setbeamercolor{block title}{fg=black,bg=red!20!white}
	\setbeamercolor{block body}{fg=black, bg=red!05!white}
}
\AfterEndEnvironment{importanttheorem}{
	\setbeamercolor{block title}{use=structure,fg=structure.fg,bg=structure.fg!20!bg}
	\setbeamercolor{block body}{parent=normal text,use=block title,bg=block title.bg!50!bg, fg=black}
}
\BeforeBeginEnvironment{theorem}{%
	\setbeamercolor{block title}{fg=white,bg=red!30!black}
	\setbeamercolor{block body}{fg=white, bg=red!10!black}
}
\AfterEndEnvironment{theorem}{
	\setbeamercolor{block title}{use=structure,fg=structure.fg,bg=structure.fg!20!bg}
	\setbeamercolor{block body}{parent=normal text,use=block title,bg=block title.bg!50!bg, fg=black}
}
\BeforeBeginEnvironment{importantproperty}{%
	\setbeamercolor{block title}{fg=black,bg=red!50!white}
	\setbeamercolor{block body}{fg=black, bg=red!30!white}
}
\AfterEndEnvironment{importantproperty}{
	\setbeamercolor{block title}{use=structure,fg=structure.fg,bg=structure.fg!20!bg}
	\setbeamercolor{block body}{parent=normal text,use=block title,bg=block title.bg!50!bg, fg=black}
}


%%%%%%%%%%%%%%%%%
\usepackage{tikz}
\usetikzlibrary{shapes,arrows}
\usetikzlibrary{positioning}
\usetikzlibrary{shapes.symbols,shapes.callouts,patterns}
\usetikzlibrary{calc,fit}
\usetikzlibrary{backgrounds}
\usetikzlibrary{decorations.pathmorphing,fit,petri}
\usetikzlibrary{automata}
\usetikzlibrary{fadings}
\usetikzlibrary{patterns,hobby}

\usepackage{pgfplots}
\pgfplotsset{compat=1.6}
\pgfplotsset{ticks=none}

\usetikzlibrary{decorations.markings}
\usetikzlibrary{arrows.meta}
\tikzset{>=stealth}

\tikzstyle{cloud} = [draw, 
ellipse,
fill=red!20, 
node distance=0.87cm,
minimum height=2em]
\tikzstyle{line} = [draw, 
-latex', 
color=yellow]


% Beginning of a section
% \AtBeginSection[]{
% 	{
% 		\setbeamercolor{background canvas}{bg=orange!10}
% 		\begin{frame}[noframenumbering,plain]
% 			\framesubtitle{\nameofthepart Chapter \insertromanpartnumber \ -- \iteminsert{\insertpart}}
% 			\tableofcontents[currentsection,currentsubsection]
% 		\end{frame}
% 	\addtocounter{page}{-1}
% 	%\addtocounter{framenumber}{-1} 
% 	}
% }


%%% SLIDES COLOURING

%\usecolortheme{owl}

\setbeamerfont{frametitle}{series=\bfseries}
\setbeamercolor{frametitle}{fg=black!05,bg=black}

\setbeamerfont{framesubtitle}{size=\normalfont\tiny}
\setbeamercolor{framesubtitle}{fg=black!05}

\setbeamercolor{background canvas}{bg=black}
\setbeamercolor{normal text}{fg=black!10}



\definecolor{bottomcolour}{rgb}{0.32,0.3,0.38}
\definecolor{middlecolour}{rgb}{0.08,0.08,0.16}
\definecolor{mycolor}{rgb}{0.4,0.4, 0.4}
% Beginning of a section
\AtBeginSection[]{
	{
		%\setbeamercolor{background canvas}[vertical shading][top=bottomcolour, middle=middlecolour, bottom=black]
		\setbeamertemplate{background canvas}[vertical shading][bottom=bottomcolour,top=black!20]
    % \setbeamertemplate{background canvas}{
    %   \begin{tikzpicture}%[remember picture,overlay]
    %     \shade[top color=yellow!75!green!33,
    %     bottom color=blue!66!green!33,
    %     middle color=blue!6!green!33]
    %   \end{tikzpicture}
    % }
    \begin{frame}[noframenumbering,plain]
			\framesubtitle{\nameofthepart Chapter \insertromanpartnumber \ -- \iteminsert{\insertpart}}
			\tableofcontents[currentsection,currentsubsection]
		\end{frame}
	\addtocounter{page}{-1}
	}
}
% Beginning of a section
\AtBeginSubsection[]{
	{
		%\setbeamercolor{background canvas}[vertical shading][top=bottomcolour, middle=middlecolour, bottom=black]
		%\setbeamertemplate{background canvas}[vertical shading][bottom=bottomcolour,top=black!20]
    \setbeamertemplate{background canvas}{
      \begin{tikzpicture}%[remember picture,overlay]
        \shade[top color=yellow!75!green!33,
        bottom color=blue!66!green!33,
        middle color=blue!6!green!33]
      \end{tikzpicture}
    }
    \begin{frame}[noframenumbering,plain]
			\framesubtitle{\nameofthepart Chapter \insertromanpartnumber \ -- \iteminsert{\insertpart}}
			\tableofcontents[currentsection,currentsubsection]
		\end{frame}
	\addtocounter{page}{-1}
	}
}

% Colours for special pages
\def\extraContent{yellow!20}

%% Allow to change slide colour
%% From: https://tex.stackexchange.com/questions/8043/change-the-background-color-of-a-frame-in-beamer
\defbeamertemplate*{background canvas}{mydefault}{%
  \ifbeamercolorempty[bg]{background canvas}{}{\color{bg}\vrule width\paperwidth height\paperheight}% copied beamer default here
}
\defbeamertemplate*{background canvas}{bg}{%
  \color{lightgray!20}\vrule width\paperwidth height\paperheight% added bg color
}
\BeforeBeginEnvironment{frame}{%
  \setbeamertemplate{background canvas}[mydefault]%
}
\makeatletter
\define@key{beamerframe}{bg}[true]{%
  \setbeamertemplate{background canvas}[bg]%
}
\makeatother
% Use with
%\begin{frame}
% \frametitle{Normal}
%\end{frame} 
%\begin{frame}[bg]
% \frametitle{With bg}
%\end{frame}


%% Vertical alignment on pages
%% From: https://tex.stackexchange.com/questions/148365/how-do-i-ask-beamer-to-exactly-fill-up-a-slide
%% Turn on with
%% \stretchon
%% (outside slide), and off with
%% \stretchoff
% \def\itemsymbol{$\blacktriangleright$}
% \let\svpar\par
% \let\svitemize\itemize
% \let\svenditemize\enditemize
% \let\svitem\item
% \let\svcenter\center
% \let\svendcenter\endcenter
% \let\svcolumn\column
% \let\svendcolumn\endcolumn
% \def\newitem{\renewcommand\item[1][\itemsymbol]{\vfill\svitem[##1]}}%
% \def\newpar{\def\par{\svpar\vfill}}%
% \newcommand\stretchon{%
%   \newpar%
%   \renewcommand\item[1][\itemsymbol]{\svitem[##1]\newitem}%
%   \renewenvironment{itemize}%
%     {\svitemize}{\svenditemize\newpar\par}%
%   \renewenvironment{center}%
%     {\svcenter\newpar}{\svendcenter\newpar}%
%   \renewenvironment{column}[2]%
%     {\svcolumn{##1}\setlength{\parskip}{\columnskip}##2}%
%     {\svendcolumn\vspace{\columnskip}}%
% }
% \newcommand\stretchoff{%
%   \let\par\svpar%
%   \let\item\svitem%
%   \let\itemize\svitemize%
%   \let\enditemize\svenditemize%
%   \let\center\svcenter%
%   \let\endcenter\svendcenter%
%   \let\column\svcolumn%
%   \let\endcolumn\svendcolumn%
% }
% \newlength\columnskip
% \columnskip 0pt



\title{Introduction to mathematical modelling of avian influenza in livestock}
\author{Julien Arino}
\date{April 2023}


\begin{document}
%\stretchon

% The title page
\begin{frame}[noframenumbering,plain]
  \titlepage
\end{frame}
\addtocounter{page}{-1}


%%%%%%%%%%%%%%%%%%%
%%%%%%%%%%%%%%%%%%%
%%%%%%%%%%%%%%%%%%%
%%%%%%%%%%%%%%%%%%%
\section{AI characteristics}

%%%%%%%%%%%%%%%%%%%
\maxFrameImage{../FIGS/Beato_etal-AI-poultry-products-cover-inverted}
\maxFrameImage{../FIGS/Beato_etal-AI-poultry-products-HPAI-inverted}
\maxFrameImage{../FIGS/Beato_etal-AI-poultry-products-LPAI-inverted}

%%%%%%%%%%%%%%%%%%%
\maxFrameImage{../FIGS/Sonnberg_etal-HPAI-natural-history-cover-inverted.png}
\maxFrameImage{../FIGS/Sonnberg_etal-HPAI-natural-history-interactions-inverted.png}
\maxFrameImage{../FIGS/Sonnberg_etal-HPAI-natural-history-periods-inverted.png}
\maxFrameImage{../FIGS/Sonnberg_etal-HPAI-natural-history-history-inverted.png}

%%%%%%%%%%%%%%%%%%%
\maxFrameImage{../FIGS/Alexander-history-AI-outbreaks-inverted}

%%%%%%%%%%%%%%%%%%%
\maxFrameImage{../FIGS/LupianiReddy-AI-history-cover-inverted}
\maxFrameImage{../FIGS/LupianiReddy-AI-history-events-inverted}
\maxFrameImage{../FIGS/LupianiReddy-AI-history-events2-inverted}
\maxFrameImage{../FIGS/LupianiReddy-AI-history-events3-inverted}

%%%%%%%%%%%%%%%%%%%
%%%%%%%%%%%%%%%%%%%
%%%%%%%%%%%%%%%%%%%
%%%%%%%%%%%%%%%%%%%
\section{Modelling AI}

\begin{frame}{Mathematical modelling of AI}
  A lot more popular than FMD!
  \vfill
  There are \emph{many} mathematical models
  \vfill
  However, most models look at zoonotic aspects, i.e., include a human component
\end{frame}

%%%%%%%%%%%%%%%%%%%
\maxFrameImage{../FIGS/Stegeman_etal-control-HPAI-models-cover-inverted}
\maxFrameImage{../FIGS/Stegeman_etal-control-HPAI-models-analytical-vs-simulation-inverted}
\maxFrameImage{../FIGS/Stegeman_etal-control-HPAI-models-transmission1-inverted}
\maxFrameImage{../FIGS/Stegeman_etal-control-HPAI-models-transmission2-inverted}

%%%%%%%%%%%%%%%%%%%
\maxFrameImage{../FIGS/Xie_etal-H9N2-within-host-cover-inverted}
\maxFrameImage{../FIGS/Xie_etal-H9N2-within-host-model-inverted}
\maxFrameImage{../FIGS/Xie_etal-H9N2-within-host-different-models-AIC-inverted}

%%%%%%%%%%%%%%%%%%%
\maxFrameImage{../FIGS/Hagenaars_etal-immune-response-AI-cover-inverted}
\maxFrameImage{../FIGS/Hagenaars_etal-immune-response-AI-flow-inverted}


%%%%%%%%%%%%%%%%%%%
\maxFrameImage{../FIGS/Liu_etal-H5N1-cover-inverted}
%\maxFrameImage{../FIGS/Liu_etal-H5N1-model-inverted}

\begin{frame}{3 types of birds $+$ environment}
  \begin{itemize}
    \item Poultry (mainly chicken), $c$
    \item Wild birds who die after H5N1 infection, $w$
    \item Wild birds who survive after H5N1 infection, $d$
    \item $V$ virus density in the environment
  \end{itemize}
\end{frame}

\begin{frame}
  \begin{center}
    \def\vertskip{*-2}
    \def\horzskip{*2.5}
    \begin{tikzpicture}[scale=1, transform shape]
      \node [rectangle, fill=gray!10, text=black] at (0\horzskip,-1\vertskip) (Sc) {$S_c$};
      \node [rectangle, fill=gray!10, text=black] at (1\horzskip,-1\vertskip) (Ic) {$I_c$};
      \node [rectangle, fill=gray!10, text=black] at (0\horzskip,0\vertskip) (Sw) {$S_w$};
      \node [rectangle, fill=gray!10, text=black] at (1\horzskip,0\vertskip) (Ew) {$E_w$};
      \node [rectangle, fill=gray!10, text=black] at (2\horzskip,0\vertskip) (Iw) {$I_w$};
      \node [rectangle, fill=gray!10, text=black] at (0\horzskip,1\vertskip) (Sd) {$S_d$};
      \node [rectangle, fill=gray!10, text=black] at (1\horzskip,1\vertskip) (Id) {$I_d$};
      \node [rectangle, fill=gray!10, text=black] at (2\horzskip,1\vertskip) (Rd) {$R_d$};
      \node [rectangle, fill=gray!10, text=black] at (3\horzskip,0\vertskip) (V) {$V$};
      %% Flows between compartments
      \path [line, very thick] (Sc.north east) to node [midway, above] (TextNode) {$\alpha_c$} (Ic.north west);
      \path [line, very thick, bend left] (Ic) to node [midway, above, sloped] (TextNode) {$r_c$} (V);
      \path [line, very thick, sloped] (Sw.north east) to node [midway, above] (TextNode) {$\alpha_{ew},\alpha_{iw}$} (Ew.north west);
      \path [line, very thick] (Ew) to node [midway, above] (TextNode) {$\mu_w$} (Iw);
      \path [line, very thick, bend left=45] (Ew) to node [midway, above, sloped] (TextNode) {$r_{ew}$} (V);
      \path [line, very thick] (Iw) to node [midway, above] (TextNode) {$r_{iw}$} (V);
      \path [line, very thick] (Sd.north east) to node [midway, above] (TextNode) {$\alpha_d$} (Id.north west);
      \path [line, very thick] (Id) to node [near end, above] (TextNode) {$\gamma_d$} (Rd);
      \path [line, very thick] (Id) to node [midway, above, sloped] (TextNode) {$r_d$} (V);
      \path [line, very thick, bend left=45] (Rd) to node [midway, below] (TextNode) {$\eta_d$} (Sd);
      %% Flows into compartments
      \path [line, very thick, dashed] (0.5\horzskip,-0.85\vertskip) to node [midway, below] (TextNode) {$\beta_{c}$} (Ic.south west);
      \path [line, very thick, dashed] (0.5\horzskip,0.15\vertskip) to node [midway, below] (TextNode) {$\beta_{w}$} (Ew.south west);
      \path [line, very thick, dashed] (0.5\horzskip,1.15\vertskip) to node [midway, below] (TextNode) {$\beta_{d}$} (Id.south west);
      %% Flows out of compartments
      \path [line, very thick] (Ic) to node [midway, right] (TextNode) {$d_{ic}$} ++(0\horzskip,-0.5\vertskip);
      \path [line, very thick] (Iw) to node [midway, right] (TextNode) {$d_{iw}$} ++(0\horzskip,-0.5\vertskip);
      \path [line, very thick] (V.north east) to node [midway, above] (TextNode) {$\delta$} ++(0.5\horzskip,0\vertskip);
      \path [line, very thick] (V.south east) to node [midway, below] (TextNode) {$d_v$} ++(0.5\horzskip,0\vertskip);
    \end{tikzpicture}    
  \end{center}  
\end{frame}



%%%%%%%%%%%%%%%%%%%
\maxFrameImage{../FIGS/MaWang-AI-seasonal-cover-inverted}

\begin{frame}{Divide year in two periods}
  \begin{itemize}
    \item During the \emph{reproductive period}, denoted $p$, poultry can reproduce
    \vfill
    \item During the \emph{overwintering period}, denoted $w$, poultry does not reproduce and AI emerges. Further subdivided
    \begin{itemize}
      \item infection phase
      \item disease-control phase
    \end{itemize}
  \end{itemize}
  \vfill
  Individuals infected with AI do not recover
\end{frame}

\begin{frame}
  \begin{align*}
  S_{n+1}^p &= f_r(N_n^w)(S_n^w+vI_n^w)+u_pS_n^w \\
  I_{n+1}^p &= u_p\sigma_pI_n^w \\
  S_{n+1} &= S_{n+1}^p\Phi\left(
    \beta\frac{I_{n}1^p}{N_{n+1}^p}
  \right) \\
  I_{n+1} &= S_{n+1}^p\left(1-\Phi\left(
    \beta\frac{I_{n}1^p}{N_{n+1}^p}
  \right)\right)+I_{n+1}^p \\
  S_{n+1}^w &= u_wS_{n+1} \\
  I_{n+1}^w &= u_w\sigma_wI_{n+1}
  \end{align*}
  where
  \begin{align*}
    N_n^p &= S_n^p+I_n^p \\
    N_n^w &= S_n^w+I_n^w \\
    N_n &= S_n+I_n 
  \end{align*}
\end{frame}

\begin{frame}{Assumptions on $\Phi$}
  $\Phi$ satisfies
  \begin{itemize}
    \item $\Phi:[0,\infty)\to[0,1]$
    \item $\Phi(0)=1$, $\lim_{x\to\infty}\Phi(x)=0$
    \item $\Phi'(x)<0$ and $\Phi''(x)\geq 0$ for all $x\in[0,\infty)$
    \item $-\Phi'(x)x<1$ for all $x\in[0,\infty)$
  \end{itemize}
\end{frame}

\begin{frame}
  They then conduct a thorough analysis of the system using 
  \[
    \R^\star = \frac{au_w}{1-u_pu_w}
  \]
  and
  \[
    \R_0=\frac{-\beta u_wu_p\sigma_p\sigma_w\Phi'(0)}{1-u_p\sigma_pu_w\sigma_w}
  \]
  \vfill
  \begin{proposition}
    Let $\R^\star>1$. If $\R_0<1$, DFE is GAS; if $\R_0>1$, DFE is unstable and system is uniformly persistent
  \end{proposition}
  Additional results (flip bifurcation, Hopf bifurcation, etc.) based on another $\hat\R_0$
\end{frame}

%%%%%%%%%%%%%%%%%%%
%\maxFrameImage{../FIGS/deJonmgHagenaars-cover-inverted}


%%%%%%%%%%%%%%%%%%%
\maxFrameImage{../FIGS/Barnes_etal-cover-inverted}

\begin{frame}{A branching process model}
  $X$ r.v. ``number of newly infected birds in a generation''
  \vfill
  Generation time: average time in days between successive generations in infection process
  \vfill
  Assume Poisson branching process for transmission, mean transmission rate $\lambda=p_L^BL+p_G^BM$
  \vfill
  Local contact probability $p_L^B$ between birds within social group of size $L$ and smaller global component between all birds within the flock of size $M$ with lower infectious contact probability $p_G^B$
\end{frame}

\begin{frame}
  Then probability generating function for number of newly infected birds in each generation is
  \[
    \Phi_X(s)=\exp\left(
      (p_L^BL+p_G^BM)(s-1)
    \right)
  \]
  and probability of extinction $q$ the smallest root in $(0,1]$ of
  \[
    q=\Phi_X(q)
  \]
  \vfill
  Reproduction number is
  \[
    R_\star = p_L^BL+p_G^BM
  \]
\end{frame}

\begin{frame}
  They then derive a model for infection within a cage
  \vfill
  And models for emergence of HPAI from LPAI
  \vfill
  Then look at various population structures (barn and free-range layer flocks, caged layer flocks, barn and free-range meat flocks, LPAI introductions)
\end{frame}

\maxFrameImage{../FIGS/Barnes_etal-enterprise-characteristics-inverted}
\maxFrameImage{../FIGS/Barnes_etal-proba-outbreak-inverted}


%%%%%%%%%%%%%%%%%%%
\maxFrameImage{../FIGS/Tiensin_etal-H5N1-Thailand-cover-inverted}
\maxFrameImage{../FIGS/Tiensin_etal-H5N1-Thailand-model-inverted}
\maxFrameImage{../FIGS/Tiensin_etal-H5N1-Thailand-characteristics-inverted}
\maxFrameImage{../FIGS/Tiensin_etal-H5N1-Thailand-estimates-inverted}
%%%%%%%%%%%%%%%%%%%
\maxFrameImage{../FIGS/Nickbakhsh_etal-cocirculation-LPAI-HPAI-cover-inverted}
\maxFrameImage{../FIGS/Nickbakhsh_etal-cocirculation-LPAI-HPAI-hypotheses-inverted}
\maxFrameImage{../FIGS/Nickbakhsh_etal-cocirculation-LPAI-HPAI-flow-inverted}
\maxFrameImage{../FIGS/Nickbakhsh_etal-cocirculation-LPAI-HPAI-model1-inverted}
\maxFrameImage{../FIGS/Nickbakhsh_etal-cocirculation-LPAI-HPAI-model2-inverted}



\end{document}
